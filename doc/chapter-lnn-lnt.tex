\chapter{Languages for Arithmetic (in construction)}

\begin{todo}
Make a chapter!
\end{todo}

Two languages built with the usual symbols of arithmetic are 
defined in ~\href{../theories/html/hydras.Ackermann.Languages.html}{Ackermann.Languages}.

\begin{itemize}
\item The first language: \texttt{LNT} (\emph{Language of Number Theory}) has just function symbols for $+$, $\times$, $0$ and successor.
\item The second language: LNN (\emph{Language of Natural Numbers})  has
the same function symbols as LNT plus one relation symbol for the strict inequality $<$ : \texttt{LT} (less than).
\end{itemize}

\subsubsection{Language of Number Theory (LNT)}

First, we declare two alphabets.

\inputsnippets{Languages/LNTDef1} 

In a second time, we build \texttt{LNT} and \texttt{LNN} by filling \texttt{Language}'s \texttt{arity} field.

\inputsnippets{Languages/LNTDef2} 

\begin{remark}
  We depart a little from \cite{Goedel}'s notations, where the 
function and relation symbols are called \texttt{Plus}, 
\texttt{Mult}, \texttt{LT}, etc. In our version, these type constructors are called \texttt{Plus\_}, 
\texttt{Mult\_}, \texttt{LT\_}, etc., while the names without final underscores are bound to term building functions (\emph{e.g.}
the function which takes two terms and builds the term representing their sum) (see Remark~\ref{rem:underscores}).
\end{remark}


\subsubsection{Language of Natural Numbers (LNN)}

\texttt{LNN} is an extension of \texttt{LNT}, by the addition 
of the $<$ relation symbol.


\inputsnippets{Languages/LNNDef}



\subsubsection{Examples}

Let us show a few examples (from ~\href{../theories/html/hydras.MoreAck.FolExamples.html}{MoreAck.FolExamples}).

\inputsnippets{FolExamples/arityTest} 
 



% \subsubsection{Examples}

% With our toy language, we can build the following terms, respectively written $a$, $f(a)$ and $f(f(v_1))$ in usual mathematical notation. 

% Writing these terms in bare \gallina syntax would result in quite clumsy scripts. Thus, we may use a still experimental notation scope which allows us to write/print terms and formulas in a 
% notation similar to \coq's.





% We put both the associated \coq{} terms, in bare \gallina{} syntax, and with a still experimental and provisional notation, defined in
% ~\href{../theories/html/hydras.Ackerm.FOL_notations.html}{MoreAck.FOL\_notations}).



% \inputsnippets{FolExamples/toyNotation}

% \inputsnippets{FolExamples/smallTerms}


For instance the term $v_1+0$, where $v_1$ is a variable,
is represented by the following \gallina term of type 
(\texttt{fol.Term LNT}).

\inputsnippets{FolExamples/v1Plus0} 






\inputsnippets{LNN/instantiations}

\section{Notations for Formulas (experimental)}

In order to get more readable terms and formulas, we can define a few notations in ~\href{../theories/html/hydras.MoreAck.FOL_notations.html}{MoreAck.FOL\_notations} and
~\href{../theories/html/hydras.MoreAck.LNN.html}{MoreAck.LNN}.
Please note that these notation scopes are experimental: We are going to use them in examples and exercises before using them in large original proof scripts (in the \texttt{ordinals/Ackermann/} sub-directory).

We try to define notation scopes as close as possible to \coq's syntax for propositions.

Let us take for instance the following proposition (in math form):

$$\forall\, v_0,\, v_0=0\vee \exists\,v_1,\,v_1=1+v_0$$

Here is a definition, using directly the \texttt{goedel/Ackermann}'s project syntax.

\inputsnippets{LNN_Examples/uglyF0}

Note that, because of redefinitions, the disjonction \texttt{orH}
can be expanded in terms of  implication and negation (for instance when we use \texttt{Compute}).

\begin{todo}
  Present the general issue about evaluation, and our provisional solution.
\end{todo}


\inputsnippets{LNN_Examples/CNNF0}
