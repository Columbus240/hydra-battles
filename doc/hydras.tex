\documentclass[twoside,a4paper]{book}

%\usepackage{fontenc}
\usepackage[utf8x]{inputenc}

\usepackage{fontspec}
%\setmainfont{DejaVu Serif}

\setmonofont{DejaVu Sans Mono}[Scale=MatchLowercase]
\usepackage{alltt}
\usepackage{multind}
\usepackage{hyperref}
\usepackage{url}
\usepackage{xcolor, fancyvrb, mdframed}
\usepackage{amsmath}
%\usepackage{fancyhdr}
\usepackage{varioref}
\usepackage{synttree, proof, centernot}
%\usepackage[firstpage]{draftwatermark}
\usepackage{verbatim}
\usepackage[note]{marginote}
\usepackage{amsmath}
\usepackage{amsfonts}
\usepackage{mathtools}
\usepackage{dsfont}
\usepackage{float}
\usepackage{threeparttable}
\usepackage{fontspec}
\usepackage{mathtools}
\usepackage{xcolor}
\usepackage{caption}
\usepackage{placeins}
\usepackage{tikz}
\usepackage{tikzsymbols}
\usetikzlibrary{arrows}
\usepackage{graphicx}
\usepackage{amsmath,mathdots,dsfont}
\usepackage{amssymb}
\usepackage{newunicodechar}
\usepackage{theorem}
\usepackage{xspace}
\usepackage{texments}
   

%%% for movies by alectryon
\usepackage{movies/snippets/assets/alectryon}
\usepackage{movies/snippets/assets/pygments}

% Prevent breaks in the middle of syntactic units
\let\OldPY\PY
\def\PY#1#2{\OldPY{#1}{\mbox{#2}}}

\setcounter{tocdepth}{1}
\definecolor{termcolor}{rgb}{0.1,0.1,0.9}
\definecolor{prooftermcolor}{rgb}{0.3,0.1,1.0}
\definecolor{metavarcolor}{rgb}{0.5,0.0,1.0}
\definecolor{darkgreen}{rgb}{0.1,0.7,0.1}
\definecolor{answercolor}{rgb}{.8,.15,.08}
\definecolor{sourcecolor}{rgb}{.07,.1,.7}
\definecolor{normalcolor}{rgb}{0.0,0.0,0.0}
\definecolor{exbluecolor}{rgb}{0.1,0.1,0.9}
\definecolor{dontlookcolor}{rgb}{0.5,0.5,0.5}
\definecolor{termcolor}{rgb}{0.05,0.05,0.4}
\definecolor{lookcolor}{rgb}{0.9,0.1,0.0}
\definecolor{darklookcolor}{rgb}{0.5,0.1,0.0}
\definecolor{prooftermcolor}{rgb}{0.3,0.1,1.0}
\definecolor{typecolor}{rgb}{1.0,0.4,0.0}
\definecolor{taccolor}{rgb}{0.1,0.10,0.0}
\definecolor{pink}{rgb}{0.8,0.6,0.6}
\definecolor{darkmagenta}{rgb}{0.4,0.0,0.6}
\definecolor{darkblue}{rgb}{0.0,0.0,0.6}


%\newcommand{\todo}{\textbf{To do}}




%%% MMCG symbols

\newcommand{\idot}[1]{\mbox{$\bullet_{#1}$}}
\newcommand{\baredot}{\mbox{$\bullet$}}
\newcommand{\iback}[1]{\mbox{$\backslash_{#1}$}}
\newcommand{\back}{\mbox{$\backslash$}}
\newcommand{\islash}[1]{\mbox{$/_{#1}$}}
\newcommand{\icomma}[3]{\mbox{$(#1\,,\,#2)_{#3}$}}
%\newcommand{\icomma}[3]{\mbox{$#1\,\mathbin{\circ}_{#3}\,#2$}}
%\newcommand{\comma}{\mbox{$\mathbin{\circ}$}}
\newcommand{\comma}{\mbox{,}}

\newcommand{\slashe}{\mbox{\textbf{$/_{\textbf{E}}$}}}
\newcommand{\slashi}{\mbox{\textbf{$/_{\textbf{I}}$}}}
\newcommand{\backe}{\mbox{\textbf{$\backslash_{\textbf{E}}$}}}
\newcommand{\backi}{\mbox{\textbf{$\backslash_{\textbf{I}}$}}}

\newcommand{\dote}{\mbox{\textbf{$\baredot_{\textbf{E}}$}}}
\newcommand{\doti}{\mbox{\textbf{$\baredot_{\textbf{I}}$}}}
\newcommand{\diame}{\mbox{\textbf{$\Diamond_{\textbf{E}}$}}}
\newcommand{\diami}{\mbox{\textbf{$\Diamond_{\textbf{I}}$}}}
\newcommand{\struct}{\mbox{\textbf{struct}}}
\newcommand{\axrule}{\mbox{\textbf{Ax}}}
\newcommand{\boxe}{\mbox{\textbf{$\Box_{\textbf{E}}$}}}
\newcommand{\boxi}{\mbox{\textbf{$\Box_{\textbf{I}}$}}}
\newcommand{\AssD}{\textbf{L$_\Diamond$}}
\newcommand{\ComD}{\textbf{P$_\Diamond$}}

\newcommand{\marginok}[1]{\marginpar{\raggedright OK:#1}}
\newcommand{\tab}{{\null\hskip1cm}}
\newcommand{\Ltac}{\mbox{{$\cal L$}tac}}
%\newcommand{\coq}{\mbox{{Coq}}}
\newcommand{\compcert}{\mbox{{CompCert}}}

\newcommand{\pcoq}{\mbox{{Pcoq}}}
\newcommand{\grail}{\mbox{{Grail}}}
\newcommand{\lcf}{\mbox{{LCF}}}
\newcommand{\hol}{\mbox{{HOL}}}
\newcommand{\pvs}{\mbox{{PVS}}}
\newcommand{\icharate}{\mbox{{Icharate}}}
\newcommand{\isabelle}{\mbox{{Isabelle}}}
%\newcommand{\coq}{\mbox{{Coq}}}
\newcommand{\prolog}{\mbox{{Prolog}}}
\newcommand{\goalbar}{\tt{}{\color{black}------------------------------------}\it}
\newcommand{\gallina}{Gallina\xspace}
\newcommand{\joker}{\texttt{\_}}
\newcommand{\eprime}{\(\e^{\prime}\)}
\newcommand{\Ztype}{\textbf{Z}}
\newcommand{\propsort}{\textbf{Prop}}
\newcommand{\setsort}{\textbf{Set}}
\newcommand{\typesort}{\textbf{Type}}
\newcommand{\ocaml}{\mbox{{OCaml}}}
\newcommand{\haskell}{\mbox{{Haskell}}}
\newcommand{\why}{\mbox{{Why}}}
\newcommand{\Pascal}{\mbox{{Pascal}}}

\newcommand{\ml}{\mbox{{ML}}}

\newcommand{\scheme}{\mbox{{Scheme}}}
\newcommand{\lisp}{\mbox{{Lisp}}}

\newcommand{\implarrow}{\mbox{$\Rightarrow$}}
\newcommand{\metavar}[1]{?#1}
\newcommand{\notincoq}[1]{#1}
\newcommand{\coqscope}[1]{\%#1}
\newcommand{\arrow}{\mbox{$\rightarrow$}}
\newcommand{\fleche}{\arrow}
\newcommand{\funarrow}{\mbox{$\Rightarrow$}}
\newcommand{\ltacarrow}{\funarrow}
 \newcommand{\coqand}{\mbox{\(\wedge\)}}
 \newcommand{\coqor}{\mbox{\(\vee\)}}
 \newcommand{\coqnot}{\mbox{\(\neg\)}}
\newcommand{\hide}[1]{}
\newcommand{\hidedots}[1]{...}
\newcommand{\sig}[3]{\texttt{\{}#1\texttt{:}#2 \texttt{|} #3\texttt{\}}}
\renewcommand{\neg}{\mbox{$\sim$}}

%%% Operateurs, etc.
\newcommand{\impl}{\mbox{$\rightarrow$}}
\newcommand{\appli}[2]{\mbox{\tt{#1 #2}}}
\newcommand{\applis}[1]{\mbox{\texttt{#1}}}
\newcommand{\abst}[3]{\mbox{\tt{fun #1:#2 \funarrow #3}}}
\newcommand{\coqle}{\mbox{$\leq$}}
\newcommand{\coqge}{\mbox{$\geq$}}
\newcommand{\coqdiff}{\mbox{$\neq$}}
\newcommand{\coqiff}{\mbox{$\leftrightarrow$}}
\newcommand{\prodsym}{\mbox{\(\forall\,\)}}
\newcommand{\exsym}{\mbox{\(\exists\,\)}}
\newcommand{\included}{\mbox{\(\subseteq\)}}

\newcommand{\substsign}{/}
\newcommand{\subst}[3]{\mbox{#1\{#2\substsign{}#3\}}}
\newcommand{\anoabst}[2]{\mbox{\tt[#1]#2}}
\newcommand{\letin}[3]{\mbox{\tt let #1:=#2 in #3}}
\newcommand{\prodep}[3]{\mbox{\tt \(\forall\,\)#1:#2,$\,$#3}}
\newcommand{\prodplus}[2]{\mbox{\tt\(\forall\,\)$\,$#1,$\,$#2}}
\newcommand{\dom}[1]{\textrm{dom}(#1)} % domaine d'un contexte (log function)
\newcommand{\norm}[1]{\textrm{n}(#1)} % forme normale (log function)
\newcommand{\coqZ}[1]{\mbox{\tt{`#1`}}}
\newcommand{\coqnat}[1]{\mbox{\tt{#1}}}
\newcommand{\coqcart}[2]{\mbox{\tt{#1*#2}}}
\newcommand{\alphacong}{\mbox{$\,\cong_{\alpha}\,$}} % alpha-congruence
\newcommand{\betareduc}{\mbox{$\,\rightsquigarrow_{\!\beta}$}\,} % beta reduction
%\newcommand{\betastar}{\mbox{$\,\Rightarrow_{\!\beta}^{*}\,$}} % beta reduction
\newcommand{\deltareduc}{\mbox{$\,\rightsquigarrow_{\!\delta}$}\,} % delta reduction
\newcommand{\dbreduc}{\mbox{$\,\rightsquigarrow_{\!\delta\beta}$}\,} % delta,beta reduction
\newcommand{\ireduc}{\mbox{$\,\rightsquigarrow_{\!\iota}$}\,} % delta,beta reduction

\newcommand{\slam}[2] {\mbox{$\lambda_#1\;#2$}}
\newcommand{\srho}[2] {\mbox{$\rho_#1\;#2$}}
\newcommand{\Vforall}[2] {\mbox{$\forall\,#1\,\in\,V(#2)$}}
\newcommand{\Eforall}[2] {\mbox{$\forall\,#1\,\in\,E(#2)$}}
\newcommand{\Vforone}[2] {\mbox{$\exists!\,#1\,\in\,V(#2)$}}
\newcommand{\Eforone}[2] {\mbox{$\exists!\,#1\,\in\,E(#2)$}}
\newcommand{\Vforsome}[2] {\mbox{$\exists\,#1\,\in\,V(#2)$}}
\newcommand{\Eforsome}[2] {\mbox{$\exists\,#1\,\in\,E(#2)$}}


% jugement de typage
\newcommand{\these}{\mbox{$\boldsymbol{\large \vdash}$}}
\newcommand{\rthese}[1]{\mbox{$\boldsymbol{\vdash_{#1}}$}}
\newcommand{\replace}[2]{\mbox{$#1[#2]$}}
\newcommand{\msubst}[2]{\mbox{$#1\{#2\}$}}
\newcommand{\disj}{\mbox{$\backslash/$}}
\newcommand{\conj}{\mbox{$/\backslash$}}
\newcommand{\deriv}[2]{\mbox{$#1\;\these\;#2$}}
\newcommand{\smalljuge}[3]{\mbox{$#1 \these #2 \boldsymbol{:} #3 $}}
%\newcommand{\juge}[3]{\mbox{$#1 \these #2 \boldsymbol{:} #3 $}}
\newcommand{\goal}[3]{\mbox{$#1,#2 \these^{\!\!\!?} #3  $}}
\newcommand{\sgoal}[2]{\mbox{$#1\,\these^{\!\!\!\!?}\, #2 $}}
\newcommand{\sequent}[2]{\mbox{$#1\;\these\; #2 $}}

\newcommand{\reduc}[5]{\mbox{$#1,#2 \these #3 \rhd_{#4}#5 $}}
\newcommand{\convert}[5]{\mbox{$#1,#2 \these #3 =_{#4}#5 $}}
\newcommand{\convorder}[5]{\mbox{$#1,#2 \these #3\leq _{#4}#5 $}}
\newcommand{\wouff}[2]{\mbox{$\emph{WF}(#1)[#2]$}}




\newcommand{\type}{\boldsymbol{:}}

% jugement absolu

%\newcommand{\ajuge}[2]{\mbox{$ \boldsymbol{\vdash} #1 : #2 $}}
\newcommand{\ajuge}[2]{\mbox{$\these #1 \boldsymbol{:} #2 $}}







\newcommand{\coqsimple}[1]{\mbox{\textbf{#1}}}
\newcommand{\cons}{\mbox{\tt\,::\,}}
\newcommand{\nil}{\mbox{\tt\,[\,]\,}}


%%% Colors 


\definecolor{mintedbgcolor}{rgb}{0.95,0.95,1.0}
\definecolor{badcoqcolor}{rgb}{0.8,0.8,0.8}
\definecolor{altcoqcolor}{rgb}{0.7,1.0,0.8}
\definecolor{mathcolor}{rgb}{0.95,0.90,0.85}
\definecolor{vertfluo}{rgb}{0.623, 0.94, 0.886}
\definecolor{lightred}{rgb}{1.0, 0.74, 0.7}
\definecolor{lightgray}{rgb}{0.7, 0.7, 0.7}
\definecolor{lookcolor}{rgb}{0.9,0.2,0.0}
\definecolor{darkred}{rgb}{.3,.0,.0}
\definecolor{sourcecolor}{rgb}{.07,.1,.7}
\definecolor{cyan}{rgb}{.0,1.0,1.0}


\newunicodechar{𝟙}{\ensuremath{\mathds{1}}}
\newunicodechar{ℤ}{\ensuremath{\mathds{Z}}}




\newcommand{\rounds}{\mbox{\,\texttt{-+->}\,}}
\newcommand{\round}{\mbox{\,\texttt{-1->}\,}}
\newcommand{\rplus}[1]{\mbox{$\,\underset{#1}{\longrightarrow}\,$}}
\newcommand{\canonseq}[2]{\mbox{$\{#1\}(#2)$}}
\newcommand{\showmath}[1]{\mathcolor{\mbox{$#1$}}}
\newcommand{\bigmath}[1]{\textcolor{blue}{\mbox{\[#1\]}}}
\newcommand{\myemph}[1]{\textcolor{lookcolor}{\,\it{#1}\,}}

%%% Coq and libraries

\newcommand{\coq}{Coq\xspace}
\newcommand{\community}{Coq-community\xspace}
\newcommand{\gaia}{Gaia\xspace}
\newcommand{\alectr}{Alectryon\xspace}
\newcommand{\equations}{Equations\xspace}
\newcommand{\Hydras}{Hydras \& Co$\text.$\xspace}
\newcommand{\HydrasLib}{Hydra-battles\xspace}
\newcommand{\gaiaHydras}{Gaia-hydras\xspace}
\newcommand{\ssreflect}{SSReflect\xspace}
\newcommand{\stdpp}{Stdpp\xspace}
\newcommand{\mathcomp}{MathComp\xspace}
\newcommand{\additions}{Addition-chains\xspace}
%%% Gaia sign

\newcommand{\mycircled}[2][none]{%
 \tikz[baseline=(a.base)]\node[draw,circle,inner sep=1pt, outer sep=0pt,fill=#1](a){\ensuremath{\scriptsize #2}\strut};
 }

{\theorembodyfont{\upshape}
 \newtheorem{exercise}{Exercise}[chapter]
 \newtheorem{project}{Project}[chapter]
\newtheorem{remark}{Remark}[chapter]
}

\newtheorem{theorem}{Theorem}[chapter]
\newtheorem{proposition}{Proposition}[chapter]

\newtheorem{lemma}{Lemma}[chapter]
\newtheorem{conjecture}{Conjecture}[chapter]
\newtheorem{definition}{Definition}[chapter]
\newtheorem{todo}{To do}[chapter]

\newmdenv[linecolor=red]{mathframe}

\newcommand{\mathcolor}[1]{\colorbox{mathcolor}{\mbox{#1}}}

%% ajoute des parenthèses légères aux expressions de la forme "f x y"

\newcommand{\kscite}[1]{\texttt{#1} {of} KS~\cite{KS81}}
\newcommand{\kpcite}[1]{\texttt{#1} {of} KP~\cite{KP82}}
\newcommand{\slcite}[1]{\texttt{#1} {of} TT~\cite{Sladek07thetermite}}

%% KS eaters
\newcommand{\gnaw}[2]{\mbox{$\{#1\}\langle #2 \rangle $}}

%\input{setup-latex}

\DefineVerbatimEnvironment%
{Coqsrc}{Verbatim}
{fontsize=\small, frame=single,rulecolor=\color{blue},fillcolor=\color{blue!05}}


\DefineVerbatimEnvironment%
{Coqanswer}{Verbatim}
{fontsize=\small, frame=single,fontshape=it,rulecolor=\color{red},fillcolor=\color{red!05}}

\DefineVerbatimEnvironment%
{Coqbad}{Verbatim}
{fontsize=\small, frame=single,rulecolor=\color{black},fillcolor=\color{black!05}}


\DefineVerbatimEnvironment%
{Coqalt}{Verbatim}
{fontsize=\small, frame=single,rulecolor=\color{green},fillcolor=\color{green!05}}

\setcounter{secnumdepth}{5}
\DeclarePairedDelimiter{\floor}{\lfloor}{\rfloor}
\DeclarePairedDelimiter{\ceil}{\lceil}{\rceil}


\newcommand{\gaiacompat}{\textcolor{blue}{Gaia}}
\newcommand{\gaiasign}{\texorpdfstring{\mycircled[orange!40]{{\mbox{\color{blue!65}{\scriptsize{G}}}}}}{Gaia-Hydra}}
\newtheorem{gaiahydra}{\gaiasign}



%%% One hypothesis per line 
\makeatletter
\renewcommand{\alectryon@hyps@sep}{\alectryon@nl}
\makeatother

%%% \snippets{A,B,C,…} inputs a series of snippets as one block (with \itemsep
%%% between them).  A, B, C should be paths to files in movies/snippets/.
\usepackage{etoolbox}
\makeatletter
\newcounter{snippets}
\newcommand{\inputsnippets}[1]
  {{\setlength{\itemsep}{1pt}\setlength{\parsep}{0pt}% Adjust spacing
    \alectryon@copymacros\begin{io}
      \forcsvlist{\item\@inputsnippet}{#1}
    \end{io}}}
\let\input@old\input % Save definition of \input
\newcommand{\@inputsnippet}[1]
  {{\renewenvironment{alectryon}{}{}% Skip \begin{alectryon} included in snippet
    \refstepcounter{snippets}%
    \input@old{movies/snippets/#1}}}
\AtEndDocument{\message{Snippets: \thesnippets}}
\makeatother

% \SetWatermarkLightness{0.8}
\definecolor{darkblue}{rgb}{0,0.1,0.7}
% \SetWatermarkScale{0.35}
% \SetWatermarkText{Work Continuously in Progress}
% \SetWatermarkColor{lightorange}


% \author{Pierre Castéran\\ Univ. Bordeaux, CNRS, Bordeaux INP, LaBRI, UMR 5800, F-33400 Talence, France\footnote{ email: pierre casteran at gmail dot com} \thanks{With contributions by Yves Bertot, \'Evelyne Contejean,  Jérémy Damour, Florian Hatat, Pascal Manoury, Karl Palmskog, Clément Pit-Claudel, and Théo Zimmermann. The formalization of primitive recursive functions was originally authored by 
% Russel O'Connor\cite{OConnor05}.}}
% \date{\today}

\title{Hydras \& Co.}
%\pagestyle{fancy}
%\fancyhead{Hydras, Ordinals, \& Co.}
\makeindex{gaiabridge}
\makeindex{coq}
\makeindex{maths}
\makeindex{hydras}
\makeindex{additions}
\makeindex{ackermann}
\begin{document}
\begin{titlepage}
   \begin{center}
       \vspace*{1cm}

       \textbf{\Huge Hydras \& Co.}

       \vspace{0.5cm}
        {\emph{\Large Formalized mathematics in \coq for inspiration and entertainement}}
            
       \vspace{1.5cm}

       \emph{Pierre Castéran, LaBRI, Univ. Bordeaux, CNRS UMR 5800\\ \textbf{email:} pierre dot casteran arobas gmail dot com. \\ With contributions by Yves Bertot, Ilm\={a}rs C\={i}rulis, \'Evelyne Contejean,  Jérémy Damour, Florian Hatat, Pascal Manoury, Karl Palmskog, Clément Pit-Claudel, and Théo Zimmermann. \\ The formalization of primitive recursive functions and Peano Arithmetic was originally authored by Russel O'Connor.}%\cite{OConnor05}.} 

 \vfill
        \today 
   
    
            
      \end{center}
\vfill
      \centering
      \includegraphics[width=13cm]{epsilon0.jpg}
      \centerline{\emph{\small {\color{darkblue}Ordinal numbers in Veblen normal form}}}
\end{titlepage}


\clearpage
\newpage
\thispagestyle{empty}
\begin{quote}
 \emph{(Aprieta el bast\'on con las dos manos, se yergue un tanto,
casi con entusiasmo)} ¡Caramba! Claro \dots{} los n\'umeros
transfinitos, Kantor \dots{}

[ Jorge Luis Borges]
\end{quote}
\vspace{17mm}
\begin{quote}
  Nessun senso percepisce l'infinito. Nessun senso permette di concludere ch'esso esista. L'infinito, in effetti, non puo' essere l'oggetto dei sensi.

  [Giordano Bruno] \emph{De l'infinito, universo e mondi}
\end{quote}

\vspace{17mm}
\begin{quote} 
I start from one point and go as far as possible. 

[John Coltrane]
\end{quote}
\vspace{6pt}
\mbox{}
\label{watercolorgray}
 {\centering
      \includegraphics[width=12cm]{hydre-pierrette.jpg}
      \centerline{\emph{``The hydra in black and white'': Watercolor  by Pierrette Cassou-Noguès}}
}
\newpage
\newpage
% \mbox{~}
\thispagestyle{empty}
\label{watercolor}
 {\centering
      \includegraphics[width=13cm]{blue-hydra.jpg}
      \centerline{\emph{``The blue hydra'': Watercolor  by Pierrette Cassou-Noguès}}
}

\newpage
% \mbox{~}
\thispagestyle{empty}
\label{roosterhydra}
 {\centering
      \includegraphics[width=12cm]{rooster-hydra.png}\\
      \centerline{\emph{``The incomplete rooster hydra'': AI art generated by Karl Palmskog}}
}

  





 \tableofcontents


%-------------------------------------------------------------------

\chapter{Introduction}

  

\vspace{16pt}

\section{Generalities}

Proof assistants are excellent tools for exploring the structure of mathematical proofs,
studying  which hypotheses are really needed, and which proof patterns are useful and/or
necessary. Since the development of a theory is represented as a bunch of computer files,
everyone is able to read the proofs with an arbitrary level of detail, or to play with the theory by writing alternate proofs or definitions.


Among all the theorems proved with the help of proof assistants like \coq{}~\cite{Coq,BC04}, \hol{}~\cite{HOL}, \isabelle{}~\cite{isabelle},  etc.,
several statements and proofs  share some interesting features:
\begin{itemize}
\item Their statements are easy to understand, even by non-mathematicians
\item Their proof requires some non-trivial mathematical tools
\item Their mechanization on computer presents some methodological interest.
\end{itemize}






This is obviously the case of the four-color theorem~\cite{fourcolors}  and the Kepler conjecture~\cite{flyspeck2015}. We do not mention impressive works like the proof of the odd-order theorem ~\cite{oddorderthm}, since understanding its statement requires a quite good mathematical culture.


In this document, we present two examples which seem to have the above properties.

\begin{itemize}
\item Hydra games (a.k.a. \emph{Hydra battles}) appear in an article published in 1982 by two mathematicians:
L. Kirby and J. Paris~\cite{KP82}: \emph{Accessible Independence Results for Peano Arithmetic}. 
Although the mathematical contents of this 
paper are quite advanced, the rules of hydra battles are very easy to understand\footnote{Let us underline the analogy between hydra battles and interactive theorem proving. Hercules is the user (you!), and hydra's heads are the subgoals: you may think that applying a tactic would solve a subgoal, but it results often in the multiplication of such tasks.}.
There are now several sites on the Internet where you can find tutorials on hydra games, together with simulators you can play with. See, for instance, the blogpost and source code written by Andrej Bauer~\cite{bauer2008,BauerHydra}.




Hydra battles, as well as Goodstein sequences~\cite{goodstein_1944, KP82}
are a nice way to present complex termination problems.
The article by Kirby and Paris presents a proof of termination
based on ordinal numbers, as well as a proof that this termination is not
provable in Peano arithmetic. In the book dedicated to 
J.P. ~Jouannaud \cite{HommageJPJ}, N.~Dershowitz and G.~Moser  give a thorough survey on this topic~\cite{Dershowitz2007}.

We present a (still partial, under continuous development) implementation in \coq of the various techniques shown in
Kirby \& Paris' and Ketonen \& Solovay's~\cite{KS81} article.

\index{gaiabridge}{Introduction}
Our library \gaiaHydras is dedicated to make compatible our lemmas with José Grimm's \gaia project (designed for \ssreflect/\mathcomp) (please look at Sect.~\vref{sect:gaia-first-intro} and the paragraphs signalled with {\gaiasign}).


\item In the second part, we are interested in computing $x^n$ with the least number of multiplications as possible. We use the notion of \emph{addition chains}~\cite{brauer1939,DBLP:journals/ipl/BerstelB87}, to generate efficient certified exponentiation functions.
\end{itemize}

\paragraph*{Warning:}

This document is \emph{not} an introductory text for \coq, and there are many aspects of this proof assistant that are not covered. 
The reader should already have some basic experience with the \coq system. The Reference Manual and several tutorials are available on the \coq website~\cite{Coq}. The first chapters of textbooks like \emph{Interactive Theorem Proving and Program Development}~\cite{BC04},
\emph{Software Foundations}~\cite{SF},
\emph{Programs and Proofs}~\cite{Sergey:PnP},
or  \emph{Certified Programming with Dependent Types} ~\cite{chlipalacpdt2011} will give you the right background.

 
 
 \subsection{Structure of \Hydras}

 \Hydras is made of three main packages: \HydrasLib, \gaiaHydras, and \additions. Figure~\ref{fig:genealogy}  illustrates
 the complex relationships: inheritance from historical contributions to \coq, and dependency with other \coq packages.
 Many thanks to  Karl Palmskog and Théo Zimmermann for the CI/CD design of \Hydras and the automation of documentation maintenance.
 Please look for a more detailed description in~\cite{jfla2022}.
 
 
 \begin{figure}[ht]
\centering
{\footnotesize
\begin{tikzpicture}[thick]

\begin{scope}[xshift=-6.5cm]
\node[rectangle, dotted,draw=black,minimum height=0.5cm,minimum width=1.5cm] (goedel) {Goedel };
\node[rectangle, dotted,draw=black,minimum height=0.5cm,minimum width=1.5cm,left of=goedel, node distance=1.9cm] (pock) { Pocklington };
\node[rectangle, dotted,draw=black,minimum height=0.5cm,minimum width=1.5cm,right of=goedel, node distance=1.8cm] (additions) { Additions };
\node[rectangle, dotted,draw=black,minimum height=0.5cm,minimum width=1.5cm,right of=additions, node distance=1.8cm] (cantor) { Cantor };

\node[rectangle, dotted,draw=black,minimum height=0.5cm,minimum width=1.5cm,right of=cantor, node distance=2.2cm] (cats) { Categories in ZFC };
\end{scope}

\begin{scope}[yshift=-1cm,xshift=-4.25cm]
  \node[rectangle, draw=black,fill=blue!20,minimum height=0.5cm,minimum width=1.5cm] (paramcoq) { Paramcoq };
\node[rectangle, draw=black,minimum height=0.5cm,minimum width=1.5cm,right of=paramcoq, node distance=5.8cm] (mathcomp) { MathComp };
\node[rectangle, draw=black,minimum height=0.5cm,minimum width=1.5cm,node distance=1.5cm,left of=paramcoq, node distance=4.5cm] (equations) { Equations };
\end{scope}

\begin{scope}[yshift=-2cm,xshift=-4cm]
\node[rectangle, draw=black,fill=orange!30,minimum height=0.5cm,minimum width=1.5cm] (hydras) { Hydra-battles };
\node[rectangle, draw=black,fill=blue!20,minimum height=0.5cm,minimum width=1.5cm,left of=hydras, node distance=3.5cm] (pockcc) { Pocklington };
\node[rectangle, draw=black,fill=orange!30,minimum height=0.5cm,minimum width=1.5cm,right of=hydras, node distance=2.5cm] (chains) { Addition-chains };
\node[rectangle, draw=black,fill=blue!20,minimum height=0.5cm,minimum width=1.5cm,right of=chains, node distance=2.2cm] (gaia) { Gaia };
\end{scope}

\begin{scope}[yshift=-3cm,xshift=-6cm]
 \node[rectangle, draw=black,fill=blue!20,minimum height=0.5cm,minimum width=1.5cm] (goedelcc) { Goedel };
 \node[rectangle, draw=black,fill=orange!30,minimum height=0.5cm,minimum width=1.5cm,right of=goedelcc, node distance=3cm] (gaiahydras) { Gaia-hydras };
\end{scope}

\draw[->,dotted] (additions.south) -- (chains.north) ;
\draw[->,dotted] (goedel) -- (goedelcc) ;
\draw[->,dotted] (goedel) -- (hydras.north west) ;
\draw[->,dotted] (cantor) -- (hydras.north east) ;
\draw[->,dotted] (cantor.south) -- (gaia.north west) ;
\draw[->,dotted] (pock) -- (pockcc) ;
\draw[->,dotted] ([xshift=2.1cm]cats) -- (gaia) ;

\draw[->] (hydras) -- (gaiahydras) ;
\draw[->] (hydras.south west) -- (goedelcc) ;
\draw[->] (gaia.south west) -- (gaiahydras.east) ;

\draw[->] (equations) -- (hydras) ;
\draw[->] (mathcomp) -- (gaia) ;
\draw[->] (mathcomp) -- (chains) ;
\draw[->] (paramcoq.south) -- (chains) ;
\draw[->] (pockcc) -- (goedelcc) ;
\end{tikzpicture}
}
\caption{Genealogy and dependencies for \Hydras packages. Dotted boxes represent historical Coq contributions, while regular boxes represent maintained \coq packages. Orange packages are maintained in the \Hydras GitHub repository, while light blue packages are maintained in other \community repositories. Dotted lines represent \coq code ancestry, while regular lines represent direct code dependencies.}
  \label{fig:genealogy}
\end{figure}

 

 
 
\subsection{Documenting theories with \alectr}

Quotations of \coq source and answers are progressively replaced from copy-pasted \emph{verbatim} to automatically generated \emph{LaTeX} blocks, using Clément Pit-Claudel's \alectr  tool~\cite{alectryonpaper, alectryongithub}.
Many thanks to Jérémy Damour, Clément Pit-Claudel  and Théo Zimmermann who designed tools for maintaining consistency between the always evolving \coq{} modules and documentation written in \emph{LaTeX}.

Besides the guarantee of consistency between theories and documentation, we hope to give a corpus for experimenting new ways of documenting the implementation of non-trivial mathematics on a proof assistant.

\subsection{Trust in our proofs}
\label{sect:trust-in-proofs}

Unlike mathematical literature, where definitions and proofs are spread out over many articles and books,
the whole proof is now inside your computer. It is composed from the \texttt{.v} files you downloaded and
parts of \coq's standard library. Thus, there is no ambiguity in our definitions and the premises of the theorems. Furthermore, you will be able to navigate through the development, using your favorite text editor or IDE, and some commands like \texttt{Search}, \texttt{Locate},  etc.



\subsection{Assumed redundancy}

It may often happen that several definitions of a given concept, or several proofs of a given theorem are possible. If all the versions present some interest, we will make them available, since each one may be of some methodological 
interest (by illustrating some tactic of proof pattern, for instance).
We use \coq's module system to make several proofs of a given theorem co-exist in our libraries (see also Sect~\vref{sect:alt-proofs}).
After some discussions of the pros and cons of each solution, we develop only one of them, leaving the others  as exercises or projects (i.e., big or difficult exercises).
In order to discuss which assumptions are really needed for proving a theorem, we will also present 
several aborted proofs.
Of course, do not hesitate to contribute nice proofs or alternative definitions!

It may also happen that some proof looks to be useless, because the proven theorem is a trivial consequence of another (also proven) result.
For instance, let us consider the three following statements:
\begin{enumerate}
\item There is no measure into $\mathbb{N}$ for proving the termination of all hydra battles (Sect~\vref{omega-case}).
\item There is no measure into the interval\,\footnote{We use the notation $[a,b)$ for denoting the set of ordinals greater or equal than $a$ and strictly less than $b$.}  $[0,\omega^2)$ for proving the termination of all hydra battles (Sect.~\vref{omega2-case}).
\item There is no measure into $[0,\mu)$ for proving the termination of all hydra battles, for any $\mu<\epsilon_0$ (Sect.\vref{sec:free-battles-case}).
\end{enumerate}

Obviously, the third theorem implies the second one, which implies the first one. So, theoretically, a library would contain only a proof of $(3)$ and remarks for $(2)$ and $(1)$. But we found it interesting to make all the three proofs available, allowing the reader to compare their common structure and notice their technical differences.
In particular, the proof of $(3)$ uses several non-trivial combinatorial properties of ordinal numbers up to $\epsilon_0$~\cite{KS81}, whilst $(1)$ and $(2)$ use simple properties of $\mathbb{N}$ and $\mathbb{N}^2$.


\subsection{About logic}

Most of the proofs we present are \emph{constructive}. Whenever possible, we provide the user with an associated function, which she or he can apply in \gallina{} or \ocaml{} in order to get a ``concrete'' feeling of the meaning of the considered theorem.
For instance, in Chapter~\vref{chap:ketonen}, the notion of \emph{limit ordinal} is
made more ``concrete'' thanks to a function \texttt{canon} which computes every item of a sequence which converges on a given limit ordinal $\alpha$. This simply typed function allows the user/reader to make her/his own experimentations.
For instance, one can very easily compute the $42$-nd item of a sequence which converges towards $\omega^{\omega^\omega}$.


 
Except in the \texttt{Schutte} library, dedicated to an axiomatic presentation of the set of countable ordinal numbers, all of our development is axiom-free, and respects the rules of intuitionistic logic. Note that we also use the \texttt{Equations} plug-in~\cite{sozeau:hal-01671777} in the definition of  several rapidly growing hierarchy of functions, in Chap.~\ref{chap:alpha-large}. This plug-in imports several known-as-harmless  axioms.

% \begin{Coqsrc}
% FunctionalExtensionality.functional_extensionality_dep : 
% forall (A : Type) (B : A -> Type) (f g : forall x : A, B x),
% (forall x : A, f x = g x) -> f = g
% \end{Coqsrc}

%\index{coq}{Commands!Print Assumptions}

At any place of our development, you may use the  \texttt{Print Assumptions {\it ident}} command in order to verify on which axiom the theorem {\it ident} may depend. The following example is extracted from 
Library~\href{../theories/html/hydras.Epsilon0.F_alpha.html}{hydras.Epsilon0.F\_alpha}, where we use the \texttt{coq-equations} plug-in (see Sect.~\vref{sect:wainer}).

\input{movies/snippets/F_alpha/DemoAssumptions}


\subsection{Typographical Conventions}


\subsubsection{Using \alectr}
Whenever possible, we use \alectr to display \coq code (definition, proof scripts) and answers.

Here are two  examples from Chapters~\ref{chapter:primrec}
and~\ref{chapter-powers}.

\vspace{4pt}


\input{movies/snippets/Ack/AckFixpointIterate.tex}

\inputsnippets{Fib2/fibEuclDemo}

\subsubsection{Verbatim quotations}

Quotations of \coq{} source  from others libraries (\coq's standard library, borrowed plug-ins)  are displayed as follows.

\begin{Coqsrc}
  Inductive CompareSpec (Peq Plt Pgt : Prop) :
  comparison -> Prop :=
    CompEq : Peq -> CompareSpec Peq Plt Pgt Eq
  | CompLt : Plt -> CompareSpec Peq Plt Pgt Lt
  | CompGt : Pgt -> CompareSpec Peq Plt Pgt Gt.
\end{Coqsrc}

We use also verbatim code inclusions when the examples lead to very long computations.

\begin{Coqbad}
Example C87_ok_slow : chain_correct 87 C87.
Proof.
Time slow_chain_correct_tac.
\end{Coqbad}

\begin{Coqanswer}
Finished transaction in 49.927 secs (49.445u,0.079s) (successful)
\end{Coqanswer}

\begin{Coqbad}
Qed.
\end{Coqbad}

 \subsection{Remark}
 
In general, we do not include full proof scripts in this document. The only exceptions are very short proofs (\emph{e.g.}, proofs by computation, or by application of automatic tactics). Likewise, we may display only the important steps on a long interactive proof, for instance, in the following lemma (\vref{lemma:L-2_6-1}):

\input{movies/snippets/Paths/Lemma261}


The reader may consult the full proof scripts with Proof General or CoqIDE, for instance.

\subsection{Active links}
The  links which appear in this pdf  document lead are of three possible kinds of destination:
\begin{itemize}
\item Local links to the document itself,
\item External links, mainly to \coq's website,
\item Local links to pages generated by \texttt{coqdoc}. According to the current makefile (through the commands \texttt{make html} and \texttt{make pdf}), 
  the pages generated by \texttt{coqdoc} are stored at 
the relative address \texttt{../theories/html/*.html} (from the location of the pdf).
Thus,  active links to our \coq{} modules may be incorrect if you did not get this \texttt{pdf} document by compiling the distribution available at
\url{https://github.com/coq-community/hydra-battles}.

\end{itemize}

\subsection{Alternative or bad definitions}
\label{sect:alt-proofs}
Finally, we decided to include definitions or lemma statements, as well as tactics,  that lead to
dead-ends or too complex developments, with the following coloring.
Bad definitions 
 are ''masked'' inside modules called \texttt{Bad}, \texttt{Bad1}, etc.

\input{movies/snippets/Schutte_basics/BadBottoma}
\input{movies/snippets/Schutte_basics/trivialProps}
\input{movies/snippets/Schutte_basics/Failure}

Likewise, alternative, but still unexplored definitions will be presented in modules
\texttt{Alt}, \texttt{Alt1}, etc. Using these definitions is left as an implicit exercise.

\input{movies/snippets/Hydra_Definitions/HydraAlt}



\section{How to install the libraries}
\label{sec:orgheadline4}
 The present distribution has been checked with version 8.14.1 of the Coq proof assistant, with a few plug-ins. \emph{Please refer to \href{https://github.com/coq-community/hydra-battle\#readme}{the README file of the project}.}


\section{Comments on exercises and projects}

Although we do not plan to include complete solutions to the exercises, 
we think it would be useful to include comments and hints, and questions/answers from the users. In contrast, ``projects'' are supposed, once completed, to be included in the repository.

Please consult the sub-directory \texttt{exercises/} of the
 project (in construction).

\section{Acknowledgements}
\label{sec:orgheadline5}
    Many thanks to Yves Bertot, Ilm\={a}rs C\={i}rulis, \'Evelyne Contejean, Jéremy Damour,   Florian Hatat,  David Ilcinkas, 
Pascal Manoury,  Karl Palmskog, Cl\'ement Pit-Claudel, Sylvain Salvati, Alan Schmitt and Théo Zimmermann for their help on the elaboration of this library and  document, and to the
 members of the \emph{Formal Methods} team and the \emph{Coq working group} at laBRI for their helpful comments 
on  oral presentations of this work. 

Many thanks also to the Coq development team and the members of the \emph{Coq Club} for interesting discussions about the \coq{} system and the Calculus of Inductive Constructions.

The author of the present document wishes to express his gratitude to the late Patrick Dehornoy, whose talk  was determinant for our desire to work on this topic.
I owe my interest in discrete mathematics and their relation to formal proofs and functional programming  to Srecko Brlek.  Equally, there is W. H. Burge's book ``\emph{Recursive Programming Techniques}'' ~\cite{burge} which was a great  source of inspiration.

Last but not least, many thanks to Pierrette Cassou-Noguès for the watercolor on pages~\pageref{watercolorgray} and
\pageref{watercolor}. Thanks to Karl Palmskog for his
\emph{rooster hydra}, page~\pageref{roosterhydra}.


\subsection{Contributions}

Yves Bertot made nice optimizations  to algorithms presented in Chapter~\ref{chapter-powers}.
\'Evelyne Contejean contributed libraries on the recursive path ordering (\emph{rpo}) for proving the well-foundedness of our representation of $\epsilon_0$ and $\Gamma_0$.
Florian Hatat proved many useful lemmas on countable sets, which we used in our adaptation of Schütte's formalization of countable ordinals. Pascal Manoury is integrating the ordinal $\omega^\omega$ into our hierarchy of ordinal notations.

The formalization of primitive recursive functions was originally a part of  Russel O'Connor's work on G\"odel's incompleteness theorems~\cite{OConnor05}. 

\label{sec:orgheadline2}

Any form of contribution  is welcome: correction of errors (typos and more serious mistakes), improvement of
Coq scripts, proposition of inclusion of new chapters, and generally any
comment or proposition that would help us. The text contains several \emph{projects} which, when completed, may improve the present work.
Please do not hesitate to share your contributions, for instance using pull requests and issues on GitHub. Thank you in advance!



% \subsubsection{Links to the Coq source}



\part{Hydras and ordinals}

\include{part-hydras-intro}
\chapter{Hydras and hydra games}

\label{sec:orgheadline91}
\label{chapter:hydras}




This chapter is dedicated to the representation of hydras and rules of the hydra game in \coq's specification language:~\gallina. 


Technically, a \emph{hydra} is just a finite ordered tree, each node of which 
has any number of sons. Contrary to the computer science tradition, we display the hydras 
with the heads up and the foot (i.e., the root of the tree) down.
Fig.~\ref{fig:Hy} represents such  a hydra, which will be referred to as \texttt{Hy} in our examples (please look at the 
module~\href{../theories/html/hydras.Hydra.Hydra_Examples.html}{Hydra.Hydra\_Examples}). 
\emph{For a less formal description of hydras, please see 
\url{https://www.smbc-comics.com/comic/hydra}.}

\begin{figure}[h]
\centering
\begin{tikzpicture}[very thick, scale=0.6]

\node (foot) at (2,0) {$\bullet$};
\node (N1) at (2,2) {$\bullet$};
\node (N2) at (2,4) {$\bullet$};
\node (N3) at (2,6) {$\bullet$};
\node (H0) at (0,2) {$\Smiley[2][vertfluo]$};
\node (H1) at (0,8) {$\Smiley[2][vertfluo]$};
\node (H2) at (4,8) {$\Smiley[2][vertfluo]$};
\node (H4) at (4,2) {$\Smiley[2][vertfluo]$};
\node (H5) at (4,4) {$\Smiley[2][vertfluo]$};
\draw (foot) -- (N1)[very thick] ;
\draw (N1) -- (N2);
\draw (N2) -- (N3);
\draw (N3) to [bend left= 10]  (H1) ;
\draw (N3) to [bend right= 16] (H2);
\draw (foot) to [bend left= 10]  (H0) ;
\draw (foot) to [bend right = 10] (H4) ;
\draw (N1) to [bend right= 16] (H5);
\end{tikzpicture}
\caption{The hydra \texttt{Hy} \label{fig:Hy}}
\end{figure}



We use a specific vocabulary for talking about hydras. Table~\ref{tab:hyd2tree} shows the correspondence between our terminology and the usual vocabulary for trees in computer science.


\begin{figure}[h]
  \centering
  \begin{tabular}{ll}
Hydras & Finite rooted trees\\
\hline
foot & root\\
head & leaf\\
node & node\\
segment  & (directed) edge \\
sub-hydra & subtree\\
daughter & immediate subtree\\
\end{tabular}
  \caption{Translation from hydras to trees}
  \label{tab:hyd2tree}
\end{figure}


The hydra \texttt{Hy} has a \emph{foot} (below), five \emph{heads}, and eight \emph{segments}. 
We leave it to the reader to define various parameters such as the height, the size, the highest arity (number of sons of a node) of a hydra. In our example, these parameters have the respective values $4$, $9$ and $3$.




\subsection{The rules of the game}

\label{sec:orgheadline44}
\label{sect:replication-def}

A \emph{hydra battle} is a fight between Hercules and the Hydra. 
More formally, a  battle is a sequence of \emph{rounds}.
At each round:
\begin{itemize}
\item If the hydra is composed of just one head, the battle is finished
and  Hercules is the winner.
\item Otherwise, Hercules chops off \emph{one} head of the hydra,

\begin{itemize}
\item If the head is at distance 1 from the foot, the head is just lost by the hydra, with no more reaction.
\item Otherwise, let us denote by \(r\) the node that was at distance \(2\) from 
the removed head in the direction of the foot,  and consider the  sub-hydra \(h'\) of \(h\), whose  root is \(r\) \footnote{$h'$ will be called ``the wounded part of the hydra'' in the subsequent text. In Figures~\vref{fig:Hy2} and ~\vref{fig:Hy4}, this sub-hydra  is displayed in red.}. Let $n$ be some natural number.
Then $h'$ is replaced by  $n+1$ of copies of \(h'\) which share the same root $r$.
 The \emph{replication factor} $n$ may be different (and generally is)   at each round of the fight.
It may be chosen by the hydra, according to its strategy, or imposed by some 
particular rule. In many presentations of hydra battles, this number is increased by $1$ at each round. In the following presentation, we will also consider battles where the hydra is free to choose its ~replication factor at each round of the battle\footnote{Let us recall that, if the chopped-off head was at distance 1 from the foot, the replication factor is meaningless.}.
\end{itemize}
\end{itemize}



Note that the description given in~\cite{KP82} of the replication process in hydra battles is also  semi-formal. 

\label{original-rules}

\begin{quote}
  ``From the node that used to be attached to the head which was just chopped off, traverse one 
segment towards the root until the next node is reached. From this node sprout $n$ replicas of 
that part of the hydra (after decapitation) which is ``above'' the segment just traversed, i.e., those 
nodes and segments from which, in order to reach the root, this segment would have to be 
traversed. If the head just chopped off had the root of its nodes, no new head is grown. ''
\end{quote}

Moreover, we note that this description is in \emph{imperative} terms. In order to formally study the properties of hydra battles, we prefer to use a mathematical vocabulary, i.e., graphs, relations, functions, etc.
Thus, the replication process will be represented as a binary relation on a data type \texttt{Hydra},
linking the state of the hydra \emph{before} and \emph{after} the transformation.
A battle will thus be represented as a sequence of terms of type \texttt{Hydra}, respecting the rules of the game.





\subsection{Example}
Let us start a battle between Hercules and the hydra \texttt{Hy} of Fig.~\ref{fig:Hy}.

At the first round, Hercules chooses to chop off the rightmost head of \texttt{Hy}.
Since this head is near the floor, the hydra simply loses this head. Let us call 
 \texttt{Hy'} the resulting state of the hydra, represented in Fig.~\vref{fig:Hy-prime}.

Next, assume Hercules chooses to chop off one of the two highest heads of \texttt{Hy'}, for instance the rightmost one. Fig.~\vref{fig:Hy2} represents the broken segment in dashed lines, and the part that will be replicated in red. Assume also that the hydra decides to add 4 copies of the red part\footnote{In other words, the replication factor at this round is equal to $4$.}. We obtain a new state \texttt{Hy''} depicted in Fig.~\ref{fig:Hy3}.



\begin{figure}[h]
\centering
\begin{tikzpicture}[very thick, scale=0.6]

\node (foot) at (2,0) {$\bullet$};
\node (N1) at (2,2) {$\bullet$};
\node (N2) at (2,4) {$\bullet$};
\node (N3) at (2,6) {$\bullet$};
\node (H0) at (0,2) {$\Smiley[2][vertfluo]$};
\node (H1) at (0,8) {$\Smiley[2][vertfluo]$};
\node (H2) at (4,8) {$\Smiley[2][vertfluo]$};
\node (H5) at (4,4) {$\Smiley[2][vertfluo]$};
%\node (H4) at (6,0) {$\Xey[2][lightgray]$};
\draw (foot) -- (N1)[very thick] ;
\draw (N1) -- (N2);
\draw (N2) -- (N3) ;
\draw (N3) to [bend left= 10]  (H1) ;
\draw (N3) to [bend right= 16] (H2);
\draw (foot) to [bend left= 10]  (H0) ;
\draw (N1) to [bend right= 16] (H5);
\end{tikzpicture}

\caption{\texttt{Hy'}: the state  of \texttt{Hy} after one round \label{fig:Hy-prime}}
\end{figure}


\begin{figure}[hp]
\centering
\begin{tikzpicture}[very thick, scale=0.5]

\node (foot) at (2,0) {$\bullet$};
\node (N1) at (2,2) {$\bullet$};
\node (N2) at (2,4)  {{\color{lightred}$\bullet$}};
\node (N3) at (2,6) {{\color{lightred}$\bullet$}};
\node (H0) at (0,2) {$\Smiley[2][vertfluo]$};
\node (H1) at (0,8) {$\Sey[2][lightred]$};
%\node (H2) at (5,0) {$\Xey[2][lightgray]$};
\node (H5) at (4,4) {$\Smiley[2][vertfluo]$};
\node (ex) at (5,8) {};
\draw (foot) -- (N1)[very thick] ;
\draw (N1) -- (N2);
\draw  (N2) -- (N3)[draw=lightred];
\draw (N3) to   [bend left= 10](H1) [draw=lightred];
\draw [dashed] (N3) to [bend left= 10](ex);
\draw (foot) to [bend left= 10]  (H0) ;
\draw (N1) to [bend right= 16] (H5);
\end{tikzpicture}
\caption{A second beheading}
\label{fig:Hy2}
\end{figure}

\begin{figure}[hp]
\centering
\begin{tikzpicture}[very thick, scale=0.6]

\node (foot) at (2,0) {$\bullet$};
\node (N1) at (2,2) {$\bullet$};
\node (N2) at (2,4) {$\bullet$};
\node (N3) at (2,6) {{\color{lightred}$\bullet$}};
\node (H1) at (0,8) {$\Smiley[2][vertfluo]$};
\node (H11) at (2,8) {$\Smiley[2][vertfluo]$};
\node (H12) at (4,8) {$\Smiley[2][vertfluo]$};
\node (H13) at (6,8) {$\Smiley[2][vertfluo]$};
\node (H14) at (8,8) {$\Smiley[2][vertfluo]$};

\node (N3) at (1,6) {$\bullet$};
\node (N31) at (2,6) {$\bullet$};
\node (N32) at (3,6) {$\bullet$};
\node (N33) at (4,6) {$\bullet$};
\node (N34) at (5,6) {$\bullet$};

\node (H0) at (0,2) {$\Smiley[2][vertfluo]$};
\node (H5) at (4,4) {$\Smiley[2][vertfluo]$};
\draw (foot) -- (N1)[very thick] ;
\draw (N1) -- (N2);
\draw (N2) -- (N3);
\draw (N2) -- (N31);
\draw (N2) -- (N32);
\draw (N2) -- (N33);
\draw (N2) -- (N34);
\draw (N3) to   [bend left= 10](H1) ;
\draw (N31) to   [bend left= 10](H11) ;
\draw (N32) to   [bend left= 10](H12) ;
\draw (N33) to   [bend left= 10](H13) ;
\draw (N34) to   [bend left= 10](H14) ;
\draw (foot) to [bend left= 10]  (H0) ;
\draw (N1) to [bend left= 10]  (H5) ;
\end{tikzpicture}
\caption{\texttt{Hy''}: the state of \texttt{Hy} after two rounds \label{fig:Hy3}}
\end{figure}

Figs.~\ref{fig:Hy4} and~\vref{fig:Hy5} represent a possible third round of the battle, with a replication factor equal to $2$. Let us call \texttt{Hy'''} the state of the hydra after that third round.

\begin{figure}[hp]
\centering
\begin{tikzpicture}[very thick, scale=0.6]

\node (foot) at (2,0)  {{\color{lightred}$\bullet$}};
\node (N1) at (2,2) {{\color{lightred}$\bullet$}};
\node (N2) at (2,4) {{\color{lightred}$\bullet$}};
\node (N3) at (2,6) {{\color{lightred}$\bullet$}};
\node (exN4) at (4,4) {};
\node (H1) at (0,8) {$\Sey[2][lightred]$};
\node (H11) at (2,8) {$\Sey[2][lightred]$};
\node (H12) at (4,8) {$\Sey[2][lightred]$};
\node (H13) at (6,8) {$\Sey[2][lightred]$};
\node (H14) at (8,8) {$\Sey[2][lightred]$};

\node (N3) at (1,6) {{\color{lightred}$\bullet$}};
\node (N31) at (2,6) {{\color{lightred}$\bullet$}};
\node (N32) at (3,6) {{\color{lightred}$\bullet$}};
\node (N33) at (4,6) {{\color{lightred}$\bullet$}};
\node (N34) at (5,6) {{\color{lightred}$\bullet$}};

\node (H0) at (0,2) {$\Smiley[2][vertfluo]$};
%\node (H5) at (4,0) {$\Xey[2][lightgray]$};
\draw (foot) -- (N1)[very thick,draw=lightred] ;
\draw (N1) -- (N2)[draw=lightred];
\draw (N2) -- (N3)[draw=lightred];
\draw (N2) -- (N31)[draw=lightred];
\draw (N2) -- (N32)[draw=lightred];
\draw (N2) -- (N33)[draw=lightred];
\draw (N2) -- (N34)[draw=lightred];
\draw (N3) to   [bend left= 10](H1) [draw=lightred];
\draw (N31) to   [bend left= 10](H11) [draw=lightred];
\draw (N32) to   [bend left= 10](H12) [draw=lightred];
\draw (N33) to   [bend left= 10](H13) [draw=lightred];
\draw (N34) to   [bend left= 10](H14) [draw=lightred];
\draw (foot) to [bend left= 10]  (H0) ;
\draw [dashed] (N1) to  [bend left= 10](exN4);
\end{tikzpicture}
\caption{A third beheading (wounded part in red) \label{fig:Hy4}}
\end{figure}

\begin{figure}[hp]
\centering
\begin{tikzpicture}[very thick, scale=0.4]

\node (foot) at (10,0) {$\bullet$};


\node (N1) at (2,2) {$\bullet$};
\node (N2) at (2,4) {$\bullet$};
\node (N3) at (2,6) {{\color{lightred}$\bullet$}};
\node (H1) at (0,8) {$\Smiley[1][vertfluo]$};
\node (H11) at (2,8) {$\Smiley[1][vertfluo]$};
\node (H12) at (4,8) {$\Smiley[1][vertfluo]$};
\node (H13) at (6,8) {$\Smiley[1][vertfluo]$};
\node (H14) at (8,8) {$\Smiley[1][vertfluo]$};

\node (N3) at (1,6) {$\bullet$};
\node (N31) at (2,6) {$\bullet$};
\node (N32) at (3,6) {$\bullet$};
\node (N33) at (4,6) {$\bullet$};
\node (N34) at (5,6) {$\bullet$};

\node (H0) at (-3,3) {$\Smiley[1][vertfluo]$};

\draw (foot) to [bend left=10] (N1)[very thick] ;
\draw (N1) -- (N2);
\draw (N2) -- (N3);
\draw (N2) -- (N31);
\draw (N2) -- (N32);
\draw (N2) -- (N33);
\draw (N2) -- (N34);
\draw (N3) to   [bend left= 10](H1) ;
\draw (N31) to   [bend left= 10](H11) ;
\draw (N32) to   [bend left= 10](H12) ;
\draw (N33) to   [bend left= 10](H13) ;
\draw (N34) to   [bend left= 10](H14) ;
\draw (foot) to [bend left = 15]  (H0) ;


% second copy 
\node (N01) at (12,2) {$\bullet$};
\node (N02) at (12,4) {$\bullet$};
\node (N03) at (12,6) {{\color{lightred}$\bullet$}};
\node (H001) at (10,8) {$\Smiley[1][vertfluo]$};
\node (H0011) at (12,8) {$\Smiley[1][vertfluo]$};
\node (H0012) at (14,8) {$\Smiley[1][vertfluo]$};
\node (H0013) at (16,8) {$\Smiley[1][vertfluo]$};
\node (H0014) at (18,8) {$\Smiley[1][vertfluo]$};

\node (N03) at (11,6) {$\bullet$};
\node (N031) at (12,6) {$\bullet$};
\node (N032) at (13,6) {$\bullet$};
\node (N033) at (14,6) {$\bullet$};
\node (N034) at (15,6) {$\bullet$};

\draw (foot) -- (N01)[very thick] ;
\draw (N01) -- (N02);
\draw (N02) -- (N03);
\draw (N02) -- (N031);
\draw (N02) -- (N032);
\draw (N02) -- (N033);
\draw (N02) -- (N034);
\draw (N03) to   [bend left= 10](H001) ;
\draw (N031) to   [bend left= 10](H0011) ;
\draw (N032) to   [bend left= 10](H0012) ;
\draw (N033) to   [bend left= 10](H0013) ;
\draw (N034) to   [bend left= 10](H0014) ;

% third copy 
\node (N001) at (22,2) {$\bullet$};
\node (N002) at (22,4) {$\bullet$};
\node (N003) at (22,6) {{\color{lightred}$\bullet$}};
\node (H001) at (20,8) {$\Smiley[1][vertfluo]$};
\node (H0011) at (22,8) {$\Smiley[1][vertfluo]$};
\node (H0012) at (24,8) {$\Smiley[1][vertfluo]$};
\node (H0013) at (26,8) {$\Smiley[1][vertfluo]$};
\node (H0014) at (28,8) {$\Smiley[1][vertfluo]$};

\node (N003) at (21,6) {$\bullet$};
\node (N0031) at (22,6) {$\bullet$};
\node (N0032) at (23,6) {$\bullet$};
\node (N0033) at (24,6) {$\bullet$};
\node (N0034) at (25,6) {$\bullet$};

\draw (foot) -- (N001)[very thick] ;
\draw (N001) -- (N002);
\draw (N002) -- (N003);
\draw (N002) -- (N0031);
\draw (N002) -- (N0032);
\draw (N002) -- (N0033);
\draw (N002) -- (N0034);
\draw (N003) to   [bend left= 10](H001) ;
\draw (N0031) to   [bend left= 10](H0011) ;
\draw (N0032) to   [bend left= 10](H0012) ;
\draw (N0033) to   [bend left= 10](H0013) ;
\draw (N0034) to   [bend left= 10](H0014) ;
\end{tikzpicture}
\caption{The configuration \texttt{Hy'''} of \texttt{Hy} \label{fig:Hy5}}
\end{figure}
\FloatBarrier

We leave it to the reader  to guess the following  rounds of the battle \dots
 % Please keep in mind that, in this 
% the hydra is free to chose any number of replications at each time, whereas
% Hercules chops only one head per round.

% Let us precise that, in this game, Hercules wins if the hydra is eventually reduced 
% to a single head. 
% We know from~\cite{KP82} that, whichever the initial configuration of the
% hydra, and the strategies of both players, Hercules eventually wins. The 
% aforementionned paper shows also that there do not exist any \emph{simple} proof of this result.


\section{Hydras and their representation in \emph{Coq}}
\label{sec:orgheadline48}

\index{hydras}{Library Hydra!Types!Hydra}
\index{hydras}{Library Hydra!Types!Hydrae}


In order to describe trees where each node can have an arbitrary (but finite) number of sons, it is usual to define a type where each node carries a \emph{forest}, \emph{i.e} a list of trees
(see for instance Chapter 14, pages 400-406 of \cite{BC04}, also available as~\cite{BC04ch14}).

For this purpose, we define two mutual \emph{ad-hoc}  inductive types, where \texttt{Hydra} is the main type, and \texttt{Hydrae} is a helper for describing finite sequences of hydras.
\label{types:Hydra}
\label{types:Hydrae}

\vspace{4pt}
\noindent
\emph{From Module~\href{../theories/html/hydras.Hydra.Hydra_Definitions.html\#Hydra}{Hydra.Hydra\_Definitions}}

\input{movies/snippets/Hydra_Definitions/HydraDef}


%\index{To do}
\index{hydras}{Projects}

\begin{project}
Look for an existing library on trees with nodes of arbitrary arity, in order to replace  this ad-hoc type with something more generic.
\end{project}


%\index{hydras}{Projects}

\begin{remark}[Mutually inductive types vs lists of hydras]
\label{hydra:mutually-inductive-vs-lists}
 Another very similar representation could use the \texttt{list} type family instead of the specific 
type \texttt{Hydrae}:


\input{movies/snippets/Hydra_Definitions/HydraAlt}

Using this representation, one can re-define all the constructions of this chapter, which is left as an exercise.
You will probably have to use patterns described for instance in~\cite{BC04} or the archives of the \coq communication channels (please consult~\url{https://coq.inria.fr/community.html}).

  
\end{remark}


\index{hydras}{Projects}

\begin{project}
  Our type \texttt{Hydra} describes hydras as
  \emph{plane oriented trees}, i.e., as drawn on a sheet of paper or computer screen. Thus, it is appropriate to consider a \emph{leftmost} or \emph{rightmost} head of
the beast. It could be interesting to consider another representation, in which
every non-leaf node has a \emph{multi-set} -- not an ordered list -- of daughters.
\end{project}

\subsubsection{Abbreviations}

We provide several notations for hydra patterns  which occur often in our developments. 

\vspace{4pt}
\noindent
\emph{From Module~\href{../theories/html/hydras.Hydra.Hydra_Definitions.html\#head}{Hydra.Hydra\_Definitions}}


\input{movies/snippets/Hydra_Definitions/headsEtc}

For instance, the hydra \texttt{Hy}  of Figure~\vref{fig:Hy} is defined as follows:

\vspace{4mm}
\noindent
\emph{From Module~\href{../theories/html/hydras.Hydra.Hydra_Examples.html\#Hy}{Hydra.Hydra\_Examples}}

\input{movies/snippets/Hydra_Examples/Hy}



Hydras quite frequently contain  multiple adjacent  copies of the same subtree. The following functions
will help us to describe and reason about replications in hydra battles.

\vspace{4pt}
\noindent
\emph{From Module~\href{../theories/html/hydras.Hydra.Hydra_Definitions.html\#hcons_mult}{Hydra.Hydra\_Definitions}}

\input{movies/snippets/Hydra_Definitions/hconsMult}



\vspace{4mm}

Let us consider for instance the hydra \texttt{Hy''} of Fig~\vref{fig:Hy3}.

\vspace{4pt}
\noindent
\emph{From Module~\href{../theories/html/hydras.Hydra.Hydra_Examples.html}{Hydra.Hydra\_Examples}}

\input{movies/snippets/Hydra_Examples/HySecond}






\subsubsection{Recursive functions on type \texttt{Hydra}}
\label{sec:orgheadline41}
\label{sec:hsize-def}




In order to  define a recursive function over the type \texttt{Hydra}, one has to consider the three constructors 
\texttt{node}, \texttt{hnil} and \texttt{hcons} of the mutually inductive types \texttt{Hydra} and \texttt{Hydrae}. 
Let us define for instance the function which  computes the number of nodes of any hydra:

\vspace{4pt}
\noindent
\emph{From Module~\href{../theories/html/hydras.Hydra.Hydra_Definitions.html}{Hydra.Hydra\_Definitions}}

\input{movies/snippets/Hydra_Definitions/hsize}

\input{movies/snippets/Hydra_Examples/HySize}


Likewise, the \emph{height} (maximum distance between the foot and a head) 
is defined by mutual recursion:

\input{movies/snippets/Hydra_Definitions/height}

\input{movies/snippets/Hydra_Examples/HyHeight}



\index{hydras}{Exercises}

\begin{exercise}
Define a function \texttt{max\_degree: Hydra $\arrow$ nat} which  returns the highest degree of a node in any hydra. For instance, the evaluation of the term \texttt{(max\_degree Hy)} should return $3$.
\end{exercise}

\subsection{Induction principles for hydras}
\label{sec:orgheadline42}


In this section, we show how induction principles are used to prove properties on the type 
\texttt{Hydra}. Let us consider for instance the following statement:
\begin{quote}
  `` The height of any hydra is strictly less than its size. ''
\end{quote}



\subsubsection{A failed attempt}

One may try to use the default tactic of proof by induction, which corresponds to an application of the automatically  generated  induction principle for  type \texttt{Hydra}:

\input{movies/snippets/Hydra_Examples/HydraInd}

Let us start a simple proof by induction.

\vspace{4pt}
\noindent
\emph{From Module~\href{../theories/html/hydras.Hydra.Hydra_Examples.html}{Hydra.Hydra\_Examples}}

\input{movies/snippets/Hydra_Examples/BadInductiona}



We might be tempted to do an induction on the sequence \texttt{s}:

\input{movies/snippets/Hydra_Examples/BadInductionb}

The first subgoal is trivial.

\input{movies/snippets/Hydra_Examples/BadInductionc}

Let us look now at the second subgoal of the induction.

\input{movies/snippets/Hydra_Examples/BadInductiond}

We notice that this sub-goal does not contain any hypothesis
on the height and size of the hydra \texttt{h}. So, it looks hard to prove the conclusion. Let's stop.

\input{movies/snippets/Hydra_Examples/BadInductione}

\subsubsection{A Principle of mutual induction}
In order to get an appropriate induction scheme for the types 
\texttt{Hydra} and \texttt{Hydrae}, we can use  \coq{}'s  command \texttt{Scheme}.


\index{coq}{Mutually inductive types}
%\index{coq}{Commands!Scheme}

\input{movies/snippets/Hydra_Definitions/HydraRect2}

\input{movies/snippets/Hydra_Examples/HydraRect2Check}





\subsubsection{A Correct proof}

Let us now use \texttt{Hydra\_rect2} for proving that the height of any hydra is strictly less than its size.
Using this scheme requires an auxiliary predicate, called \texttt{P0} in \texttt{Hydra\_rect2}'s statement. 

\vspace{4pt}
\noindent
\emph{From Module~\href{../theories/html/hydras.Hydra.Hydra_Definitions.html}{Hydra.Hydra\_Definitions}}

\input{movies/snippets/Hydra_Definitions/hForall}

\emph{From Module~\href{../theories/html/hydras.Hydra.Hydra_Examples.html}{Hydra.Hydra\_Examples}}
\input{movies/snippets/Hydra_Examples/heightLtSizea}




The first subgoal is easily solvable, using some arithmetic.
The second and third ones are
almost trivial. We let the reader look at the source. 
\input{movies/snippets/Hydra_Examples/heightLtSizez}



\index{hydras}{Exercises}

\begin{exercise}
It happens very often that, in the proof of  a proposition of the form 
(\texttt{$\forall\,$ h:Hydra, $P$ h}), the predicate \texttt{P0}
is  (\texttt{h\_forall $P$}).  Design a tactic for induction on hydras that frees the user from binding explicitly \texttt{P0},  and solves trivial subgoals. Apply it for writing  a shorter proof script of \texttt{height\_lt\_size}.
\end{exercise}
 


\section{Relational description of hydra battles}


In this section, we represent the rules of hydra battles as a binary relation associated with
a \emph{round}, i.e., an interaction composed of the two following actions:
\begin{enumerate}
\item Hercules chops off one head of the hydra.
\item Then, the  hydra replicates the wounded part (if the head is at distance $\geq 2$ from the foot).
\end{enumerate}
The relation associated with each round of the battle is parameterized  by the \emph{expected} replication  factor (irrelevant if the chopped head is at distance 1 from the foot,
but present for consistency's sake).

In our description,  we will apply the following naming convention: if $h$ represents the configuration of the hydra before a round, then the configuration of $h$ after this round will be called $h'$.
 Thus, we are going to define a proposition  (\texttt{round\_n $n\;h\;h'$})  whose intended meaning will be `` the hydra $h$  is transformed into $h'$  in a single round of a battle, with the expected replication factor $n$ ''.


Since the replication of parts of the hydra depends on the distance of the chopped head from  the foot, we  decompose our description into two main  cases, under the form of a bunch of [mutually] inductive predicates over the types \texttt{Hydra} and \texttt{Hydrae}.

The mutually exclusive cases we consider are the following:
\begin{itemize}
\item \textbf{R1}: The chopped off head was at distance 1 from the foot.
\item \textbf{R2}: The chopped off head was at a distance greater than or equal to  $2$ from the foot.
\end{itemize}



\subsection{Chopping off a head at distance 1 from the foot (relation  R1)}

If Hercules chops off a head close to the root, there is no replication at all. We use an auxiliary 
predicate \texttt{S0}, associated with the removing of one head from a sequence of hydras.


\vspace{4pt}\emph{From Module\href{../theories/html/hydras.Hydra.Hydra_Definitions.html}{Hydra.Hydra\_Definitions}}

\input{movies/snippets/Hydra_Definitions/S0Def}
\input{movies/snippets/Hydra_Definitions/R1Def}

\subsubsection{Example}
\label{sec:orgheadline45}

Let us represent in \coq{}   the transformation of the hydra of Fig.~\vref{fig:Hy} into
the configuration represented in Fig.~\ref{fig:Hy-prime}.

\vspace{4pt}
\emph{From Module~\href{../theories/html/hydras.Hydra.Hydra_Examples.html}{Hydra.Hydra\_Examples}}


\input{movies/snippets/Hydra_Examples/Hy1}


\subsection{Chopping off a head at distance \texorpdfstring{$\geq 2$}{>= 2} from the foot (relation R2) }


Let us now consider beheadings  where the chopped-off head is at distance greater than or equal to $2$ from the foot. All the following relations are parameterized by the replication factor  $n$.

 Let $s$ be a sequence of hydras. 
The proposition (\texttt{S1 n s s'}) holds if $s'$ is obtained by replacing some element $h$ of $s$ by 
$n+1$ copies of $h'$, where  the proposition (\texttt{R1 h h'}) holds, in other words, $h'$ is just $h$, without the chopped-off  head. \texttt{S1} is an inductive relation with two constructors that allow us to choose the position in $s'$ of the wounded sub-hydra $h$.

\vspace{4pt}
\noindent
\emph{From Module~\href{../theories/html/hydras.Hydra.Hydra_Definitions.html\#S1}{Hydra.Hydra\_Definitions}}

\input{movies/snippets/Hydra_Definitions/S1Def}


The rest of the definition is composed of two mutually inductive relations on hydras and sequences of hydras. The first constructor of \texttt{R2} describes the case where the chopped head is exactly at height $2$. The others constructors allow us to consider beheadings at height strictly greater than $2$.


\vspace{4pt}
\emph{From Module~\href{../theories/html/hydras.Hydra.Hydra_Definitions.html\#R2}{Hydra.Hydra\_Definitions}}

\input{movies/snippets/Hydra_Definitions/R2Def}

\subsubsection{Example}
Let us prove the transformation of \texttt{Hy'} into \texttt{Hy''} (see Fig.~\vref{fig:Hy3}). We use an experimental set of tactics\footnote{See the \texttt{Ltac} definitions in~\href{../theories/html/hydras.Hydra.Hydra_Definitions.html\#R2}{Hydra.Hydra\_Definitions.}}  for specifying the place where the 
interaction between Hercules and the hydra holds. 


\vspace{4pt}\emph{From Module~\href{../theories/html/hydras.Hydra.Hydra_Examples.html}{Hydra.Hydra\_Examples}}. 

\input{movies/snippets/Hydra_Examples/R2Example}


The reader is encouraged to look at all the successive subgoals of this example.
\emph{Please consider also exercise~\vref{exo:interactive-battle}.}


\subsection{Binary relation associated with a round}

Let us merge \texttt{R1} and \texttt{R1}  into a single relation.
First,  we define the  relation \texttt{(round\_n n h h')} where \texttt{n} is the expected number of  replications (irrelevant in the case of an \texttt{R1}-transformation).
Then, we define a \emph{round} (small step) of a battle
by abstraction over \texttt{n}, 

\vspace{4pt}
\emph{From Module~\href{../theories/html/hydras.Hydra.Hydra_Definitions.html\#round_n}{Hydra.Hydra\_Definitions}}

\index{hydras}{Library Hydra!Predicates!round\_n}
\index{hydras}{Library Hydra!Predicates!round}
\label{sect:infix-round}

\inputsnippets{Hydra_Definitions/roundNDef, Hydra_Definitions/roundDef}





\index{hydras}{Projects}

\begin{project}
Give a direct translation of Kirby and Paris's description of hydra battles (quoted on page~\pageref{original-rules}) and prove that our relational description is consistent with theirs.
\end{project}


\subsection{Rounds and battles}


Using library \href{https://coq.inria.fr/distrib/current/stdlib/Coq.Relations.Relation_Operators.html}{Relations.Relation\_Operators}, we define \texttt{round\_plus},  the transitive closure of \texttt{round}, and \texttt{round\_star},  the reflexive and transitive closure of \texttt{round}.

\label{sect:infix-rounds} 

\input{movies/snippets/Hydra_Definitions/roundPlus}

\index{hydras}{Exercises}

\begin{exercise}
  Prove that if \texttt{$h$ -+-> $h'$}, then
  the height of $h'$ is less or equal than the height of $h$.

\end{exercise}

\begin{remark}
\label{remark:transitive-closure}
\coq's library \href{https://coq.inria.fr/distrib/current/stdlib/Coq.Relations.Relation_Operators.html}{Coq.Relations.Relation\_Operators} 
contains three logically equivalent definitions of the transitive closure of a binary relation. This equivalence is proved in 
\href{https://coq.inria.fr/distrib/current/stdlib/Coq.Relations.Operators_Properties.html}{Coq.Relations.Operators\_Properties} . 

Why three definitions for a single mathematical concept?
Each definition generates an associated induction principle. 
 According to the form of statement one would like to prove, there is a ``best choice'':

\begin{itemize}
\item To prove $\forall y, x\,R^+\,y \;\arrow\; P\,y$, prefer 
\texttt{clos\_trans\_n1}
\item To prove proving $\forall x,\,x\,R^+\,y \;\arrow\; P\,x$, prefer \texttt{clos\_trans\_1n}
\item To prove $\forall x\,y, \,x\,R^+\,y \;\arrow\;P\,x\,y$,  
prefer \texttt{clos\_trans},
\end{itemize}
But there is no ``wrong choice'' at all: the equivalence lemmas in \linebreak 
\href{https://coq.inria.fr/distrib/current/stdlib/Coq.Relations.Operators_Properties.html}{Coq.Relations.Operators\_Properties} 
 allow the user
to convert any one of the three closures into another one before applying the corresponding elimination tactic.
The same remark also holds for reflexive and transitive closures. 
\end{remark}

\index{hydras}{Exercises}

\begin{exercise}
Define a restriction of \coqsimple{round},  where Hercules always chops off
the leftmost among the lowest heads.

Prove that, if $h$ is not a simple head, then there exists a unique $h'$ such that $h$  is transformed into $h'$ in one round, according to this restriction.


\end{exercise}

\index{hydras}{Exercises}

\begin{exercise}[Interactive battles]
\label{exo:interactive-battle}
Given a hydra \texttt{h}, the specification of a hydra battle for \texttt{h} is the type 
\Verb@{h':Hydra | h -*-> h'}@. In order to avoid long sequences of \texttt{split}, \texttt{left}, and 
\texttt{right}, design a set of dedicated tactics for the interactive building of a battle.
Your tactics will have the following functionalities:
\begin{itemize}
\item  Choose to stop a battle, or continue
\item Choose an expected number of replications
\item Navigate in a hydra, looking for a head to chop off.
\end{itemize}

Use your tactics for simulating a small part of a hydra battle, for instance the rounds which lead from
\texttt{Hy} to \texttt{Hy'''}  (Fig.~\vref{fig:Hy5}).

\textbf{Hints:} 
\begin{itemize}

\item Please keep in mind that the last  configuration of your interactively built battle is known only at the end of the battle. Thus, you will have to create and solve subgoals with existential variables. For that purpose, the tactic \texttt{eexists}, applied to the 
goal \Verb@{h':Hydra | h -*-> h'}@ generates the subgoal \Verb|h -*-> ?h'|.
\item You may use Gérard Huet's \emph{zipper} data structure~\cite{zipper} for writing tactics associated with Hercules's  interactive search for a head to chop off.
\end{itemize}






\end{exercise}




\subsection{Classes of battles}
\label{sect:battle-classes}

In some presentations of hydra battles, e.g.~\cite{KP82, bauer2008}, the transformation associated with the $i$-th round may depend on $i$. For instance, in these articles, the replication factor at the $i$-th round is equal to $i$. In other examples, one can allow the hydra to apply any replication factor at any time. In order to be the most general as possible, we define the type of predicates which relate the state of the hydra before and after the $i$-th round of a battle.

\vspace{6pt}
\noindent
\emph{From Module~\href{../theories/html/hydras.Hydra.Hydra_Definitions.html}{Hydra.Hydra\_Definitions}}
\label{types:Battle}
\index{hydras}{Library Hydra!Type classes!Battle}

\input{movies/snippets/Hydra_Definitions/BattleDef}

The most general class of battles is \texttt{free}, which allows the hydra to choose any replication factor at every step:

\vspace{6pt}
\noindent
\emph{From Module~\href{../theories/html/hydras.Hydra.Hydra_Definitions.html\#free}{Hydra.Hydra\_Definitions}}

\input{movies/snippets/Hydra_Definitions/freeDef}

We chose to call \emph{standard}\footnote{This appellation is ours. If there is a better one, we will change it.} the kind of battles which appear  most often in the literature and correspond to an arithmetic progression of the replication factor : $0,1,2,3, \dots$


\vspace{6pt}
\noindent
\emph{From Module~\href{../theories/html/hydras.Hydra.Hydra_Definitions.html\#standard}{Hydra.Hydra\_Definitions}}
\input{movies/snippets/Hydra_Definitions/standardDef}



\subsection{Big steps}

Let $B$ be any instance of class \texttt{Battle}. It is easy to define inductively the relation between the $i$-th and the $j$-th steps of a battle of type $B$. 

\vspace{6pt}
\noindent\emph{From Module~\href{../theories/html/hydras.Hydra.Hydra_Definitions.html\#fight}{Hydra.Hydra\_Definitions}}

\input{movies/snippets/Hydra_Definitions/battleRelDef}

The following property allows us to build battle rounds by composition of smaller ones.

%% TODO display subgoals when fixed

\vspace{6pt}
\noindent\emph{From Module~\href{../theories/html/hydras.Hydra.Hydra_Lemmas.html}{Hydra.Hydra\_Lemmas}}
\noindent
\input{movies/snippets/Hydra_Lemmas/battleTrans}


\section{A long battle}
\label{sect:big-battle}


In this section we consider a simple example of battle, starting with a small hydra,
shown on figure~\vref{fig:hinit}, with a simple strategy for both players:

\begin{itemize}
\item At each round, Hercules chops off the rightmost head of the hydra.
\item The battle is {standard}: at the round number $i$, the expected replication factor is $i$.
\end{itemize}



\begin{figure}[h]
  \centering
  \begin{tikzpicture}[thick, scale=0.30]
 \node (foot) at (6,0) {$\bullet$};
\node (n1) at  (3,3) {$\bullet$};
\node (h1) at  (1,6) {$\Smiley[1][green]$};
\node (h2) at  (3,6) {$\Smiley[1][green]$};
\node (h3) at  (6,6) {$\Smiley[1][green]$};
\node (h4) at  (6,6) {$\Smiley[1][green]$};
\node (h5) at  (6,3) {$\Smiley[1][green]$};
\node (h6) at  (9,3) {$\Smiley[1][green]$};
\draw (foot) -- (n1);
\draw (n1) to   [bend left=20] (h1);
\draw (n1) to   (h2);
\draw (n1) to   [bend right=20] (h3);
\draw (foot) -- (h5);
\draw (foot) to  [bend right=20] (h6);
\end{tikzpicture}

  \caption{The hydra \texttt{hinit}}
  \label{fig:hinit}
\end{figure}

\emph{From Module~\href{../theories/html/hydras.Hydra.BigBattle.html}{Hydra.BigBattle}}

\input{movies/snippets/BigBattle/hinitDef}


The lemma we would like to prove is ``The considered battle lasts exactly $N$ rounds'',
with $N$ being a natural number we have to guess.

But the  battle is so long that no \emph{test} can give us any estimation of its length. Nevertheless, in order to  guess this length, we made some experiments, computing with \gallina{}, \coq{}'s  functional programming language.
Thus, we can consider this development as a collaboration of proof with computation.
In the rest of this section, we show how we found experimentally the value of the number $N$.
The complete proof is in file \url{../theories/html/hydras.Hydra.BigBattle.html}. 

\subsection{First rounds}
During the two first rounds, our hydra loses its two rightmost heads.  Figure~\vref{fig:hinit-plus2} shows the state of the hydra   just before the third round.


\begin{figure}[h]
  \centering
  \begin{tikzpicture}[thick, scale=0.30]
 \node (foot) at (3,0) {$\bullet$};
\node (n1) at  (3,3) {$\bullet$};
\node (h1) at  (1,6) {$\Smiley[1][green]$};
\node (h2) at  (3,6) {$\Smiley[1][green]$};
\node (h3) at  (6,6) {$\Smiley[1][green]$};
\node (h4) at  (6,6) {$\Smiley[1][green]$};
\draw (foot) -- (n1);
\draw (n1) to   [bend left=20] (h1);
\draw (n1) to   (h2);
\draw (n1) to   [bend right=20] (h3);
\end{tikzpicture}

  \caption{The hydra (\texttt{hyd1 h3})}
  \label{fig:hinit-plus2}
\end{figure}

The following lemma  is a formal description of these first rounds, in terms of the \texttt{rounds} predicate.


\input{movies/snippets/BigBattle/L02}


\subsection{Looking for regularities}


A first study with pencil and paper suggested us that, after three rounds, the hydra always looks like in figure~\vref{fig:hinit-plusn} (with a variable number of 
subtrees of height 1 or 0).
Thus, we introduce a few handy abbreviations.

\input{movies/snippets/BigBattle/Notations}


For instance, the hydra shown in  Fig~\ref{fig:hinit-plusn} is  (\texttt{hyd 3 4 2}). The hydra (\texttt{hyd 0 0 0})  is the ``final'' hydra of any terminating battle, {i.e.},
a tree whith exactly one node and no edge.

\begin{figure}[h]
  \centering
  \begin{tikzpicture}[thick, scale=0.30]
 \node (foot) at (15,0) {$\bullet$};
\node (a) at  (3,4) {$\bullet$};
\node (b) at  (6,4) {$\bullet$};
\node (c) at  (9,4) {$\bullet$};
\node (d) at  (13,4) {$\bullet$};
\node (e) at  (16,4) {$\bullet$};
\node (f) at  (19,4) {$\bullet$};
\node (g) at  (22,4) {$\bullet$};
\node (h) at  (25,4) {$\Smiley[1][green]$};
\node (i) at  (28,4) {$\Smiley[1][green]$};
\node (aa) at  (2.5,8) {$\Smiley[1][green]$};
\node (ab) at  (3.5,8) {$\Smiley[1][green]$};
\node (ba) at  (5.5,8) {$\Smiley[1][green]$};
\node (bb) at  (6.5,8) {$\Smiley[1][green]$};
\node (ca) at  (8.5,8) {$\Smiley[1][green]$};
\node (cb) at  (9.5,8) {$\Smiley[1][green]$};
\node (da) at  (13,8) {$\Smiley[1][green]$};
\node (ea) at  (16,8) {$\Smiley[1][green]$};
\node (fa) at  (19,8) {$\Smiley[1][green]$};
\node (ga) at  (22,8) {$\Smiley[1][green]$};
\draw (foot) -- (a);
\draw (foot) -- (b);
\draw (foot) -- (c);
\draw (foot) -- (d);
\draw (foot) -- (e);
\draw (foot) -- (f);
\draw (foot) -- (g);
\draw (foot) -- (h);
\draw (foot) -- (i);
\draw (a) -- (aa);
\draw (a) -- (ab);
\draw (b) -- (ba);
\draw (b) -- (bb);
\draw (c) -- (ca);
\draw (c) -- (cb);
\draw (d) -- (da);
\draw (e) -- (ea);
\draw (f) -- (fa);
\draw (g) -- (ga);
\end{tikzpicture}

  \caption{The hydra (\texttt{hyd 3 4 2})}
  \label{fig:hinit-plusn}
\end{figure}


With these notations, we get a formal description of the first three rounds.

\input{movies/snippets/BigBattle/L23L03}


\subsection{Testing  \dots}
\label{sect:testing}
In order to study \emph{experimentally} the different  configurations of the  battle, we will use a simple data type for representing the states as tuples composed of
the round number, and the respective number of daughters  \texttt{h2}, \texttt{h1}, and heads
of the current hydra.


\input{movies/snippets/BigBattle/stateDef}



The following function returns the next configuration of the game.
Note that this function is defined only for making experiments and is not  ``certified''.  Formal proofs about our battle only start with the lemma
\texttt{step\_rounds}, page~\pageref{lemma:step-battle}.

\input{movies/snippets/BigBattle/nextDef}



We can make bigger steps through iterations of \texttt{next}.
The functional \texttt{iterate}, similar to Standard Library's \texttt{Nat.iter},
is defined and studied in~\href{../theories/html/hydras.Prelude.Iterates.html\#iterate}{Prelude.Iterates}.
\index{hydras}{Library Prelude!iterate}

\label{Functions:iterate}

\input{movies/snippets/Iterates/iterateDef}


The following function computes the state of the battle at the $n$-th round.

\input{movies/snippets/BigBattle/testDefTests}

The battle we are studying looks to be awfully long. Let us concentrate our
tests on some particular events : the states where $\texttt{nh}=0$.
From the value of \texttt{test 5},  it is obvious that at the 10-th round, the counter \texttt{nh} is equal to zero.


\input{movies/snippets/BigBattle/smartTest}

Thus, $(1 + 11)$ rounds later, the \texttt{n1} field is equal to $2$, and 
\texttt{nh}   to $0$. 

\input{movies/snippets/BigBattle/smartTestb}



Next round, we decrement \texttt{n2} and set \texttt{n1} to $95$.

\input{movies/snippets/BigBattle/smartTestc}



We now have some intuition of the sequence.
It looks like the next ``\texttt{nh}=0'' event will happen at the $192=2(95+1)$-th round, then at the $2(192+1)$-th round, etc.

\input{movies/snippets/BigBattle/doubleS}


\subsection{Proving \dots}
We are now able to reason about the sequence of transitions defined by our hydra battle. 

Let us define a binary relation associated with every round of the battle.
In the following definition \texttt{i} is associated with the round number (or date, if we consider a discrete time), and \texttt{a}, \texttt{b}, \texttt{c} respectively associated with the number of occurrences of \texttt{h2}, \texttt{h1} and heads connected to the hydra's foot. For convenience, we do not use the type \texttt{state} of the preceding section, but consider the round numbers and the number of hydras \texttt{h2}, \texttt{h1} and heads as separate arguments of the relation (which is no more ---formally--- ``binary'').

\input{movies/snippets/BigBattle/oneStep}

The relation between \texttt{one\_step} and the rules of hydra battles is asserted by the following lemma. 

\label{lemma:step-battle}

\input{movies/snippets/BigBattle/stepBattle}

\vspace{4pt}

Next, we define ``big steps'' as the transitive closure of \texttt{one\_step},
and reachability (from the initial configuration of figure~\ref{fig:hinit} at time $0$).


\input{movies/snippets/BigBattle/steps}



The following lemma establishes a relation between \texttt{steps} and the predicate \texttt{battle}.

\input{movies/snippets/BigBattle/stepsBattle}

\vspace{4pt}

Thus, any result about \texttt{steps} will be applicable to standard battles.
Using the predicate \texttt{steps},  our study of the length of the considered battle
can  be decomposed into three parts:

\begin{enumerate}
\item  Characterization and proofs of regularities of some events (inspired by our experiments of Sect.~\ref{sect:testing}).
\item Study of the beginning of the battle
\item Computing the exact length of the battle.
\end{enumerate}

First, we prove that, if at round $i$ the hydra is equal to
(\texttt{hyd a (S b) 0}), then it will be equal to (\texttt{hyd a b 0}) at the $2(i+1)$-th round.

\vspace{4pt}


\input{movies/snippets/BigBattle/LS}

From now on, the lemma \texttt{reachable\_S} allows us to watch larger and larger steps of 
the battle.



\input{movies/snippets/BigBattle/L4}

\input{movies/snippets/BigBattle/L10To95}

\subsection{Giant steps}

We are now able to make bigger steps in the simulation of the battle.
First, we iterate the lemma \texttt{reachable\_S}.


\vspace{4pt}

\input{movies/snippets/BigBattle/Bigstep}

\vspace{4pt}

Applying lemmas \texttt{BigStep} and \texttt{L95} we make a first jump.

\vspace{4pt}

\input{movies/snippets/BigBattle/MDef}



Figure~\ref{fig:HM}  represents the hydra at the $M$-th round.
At the $(M+1)$-th round, it will look like in fig~\ref{fig:HM-plus1}.





\begin{figure}[htb]
\centering
\begin{tikzpicture}[very thick, scale=0.5]
\node (foot) at (2,0) {$\bullet$};
\node (N1) at (2,2) {$\bullet$};
\node (N2) at (3,4) {$\Smiley[2][green]$};
\node (N3) at (1,4) {$\Smiley[2][green]$};
\draw (foot) -- (N1);
\draw (N1) to [bend right =15] (N2);
\draw (N1) to  [bend left=15](N3);
\end{tikzpicture}
\caption{\label{fig:HM}}
The state of the hydra after $M$ rounds.
% The hydra \texttt{h} of the proof that \(\omega^2\) is too small for proving Hercules' victory

\end{figure}


\begin{figure}[htb]
\centering
\begin{tikzpicture}[very thick, scale=0.5]
\node (foot) at (10,0) {$\bullet$};
\node (N1) at (0,5) {$\bullet$};
\node (N12) at (0,8) {$\Smiley[2][green]$};
\node (N2) at (2,5) {$\bullet$};
\node (N22) at (2,8) {$\Smiley[2][green]$};
\node (N3) at (4,5) {$\bullet$};
\node (N32) at (4,8) {$\Smiley[2][green]$};
\node (N4) at (6,5) {$\bullet$};
\node (N42) at (6,8) {$\Smiley[2][green]$};

\node (Ndots) at (12,8) {\Huge $\dots$};
\node (Ndots2) at (12,5) {\Huge $\dots$};

\node (N8) at (18,5) {$\bullet$};
\node (N82) at (18,8) {$\Smiley[2][green]$};
\node (N9) at (20,5) {$\bullet$};
\node (N92) at (20,8) {$\Smiley[2][green]$};


\draw (foot) -- (N1);
\draw (foot) -- (N2);
\draw (foot) -- (N3);
\draw (foot) -- (N4);
\draw (foot) -- (N8);
\draw (foot) -- (N9);
\draw (N1) to  (N12);
\draw (N2) to  (N22);
\draw (N3) to  (N32);
\draw (N4) to  (N42);
\draw (N8) to  (N82);
\draw (N9) to  (N92);
\end{tikzpicture}
\caption{\label{fig:HM-plus1}}
The state of the hydra after $M+1$ rounds (with $M+1$ heads). 

\end{figure}


\input{movies/snippets/BigBattle/L295S}

\vspace{4pt}

Then, applying once more the lemma \texttt{BigStep}, we get the exact time when
Hercules wins!

\vspace{4pt}

\input{movies/snippets/BigBattle/NDef}

\vspace{4pt}

We are now able to prove formally that the considered battle is 
composed of $N$ steps.

\vspace{4pt}

\input{movies/snippets/BigBattle/Done}


\subsection{A minoration lemma}

Now, we would like to get an intuition of  how big the number $N$ is.
For that purpose, we use a minoration of the function \texttt{doubleS} by the
function (\texttt{fun n => 2 * n}).

\vspace{4pt}

\input{movies/snippets/Exp2/exp2Def}

\vspace{4pt}

Using a few facts (proven in 
\href{../theories/html/hydras.Hydra.BigBattle.html}{hydras.Hydra.BigBattle}),we get several  minorations.

\input{movies/snippets/BigBattle/minorationLemmas}

\vspace{4pt}


The number $N$ is greater than or  equal to $2^{2^{95}\times 95}.$ If we wrote $N$ in base $10$, $N$ would require at least $10^{30}$ digits!


\section{Generic properties}


The example we just studied shows that the termination of any battle may take a very long time. If we want to study hydra battles in general, we have to consider 
any hydra and any strategy, both for Hercules and the hydra itself. So, we first  give some definitions, generally borrowed from transition systems vocabulary (see~\cite{tel_2000} for instance).


\subsection{Liveliness}


Let $B$ be an instance of \texttt{Battle}. We say that $B$ is \emph{alive} if
for any configuration $(i,h)$, where $h$ is not a head, there exists a further step in class $B$.


\vspace{4pt}
\emph{From Module~\href{../theories/html/hydras.Hydra.Hydra_Definitions.html\#Alive}{Hydra.Hydra\_Definitions}}

\input{movies/snippets/Hydra_Definitions/AliveDef}


The theorems \texttt{Alive\_free} and \texttt{Alive\_standard} of the module 
\href{../theories/html/hydras.Hydra.Hydra_Theorems.html}{Hydra.Hydra\_Theorems} show that the classes \texttt{free} and \texttt{standard} satisfy this property.

\vspace{4pt}
\emph{From Module~\href{../theories/html/hydras.Hydra.Hydra_Lemmas.html\#next_round_dec}{Hydra.Hydra\_Lemmas}}

\input{movies/snippets/Hydra_Lemmas/nextRoundDec}

\emph{From Module~\href{../theories/html/hydras.Hydra.Hydra_Theorems}{Hydra.Hydra\_Theorems}}

\input{movies/snippets/Hydra_Theorems/AliveThms}



\subsection{Termination}

The termination of \emph{any}  battle is naturally expressed by the predicate \texttt{well\_founded} defined in the module \href{https://coq.inria.fr/distrib/current/stdlib/Coq.Init.Wf.html}{Coq.Init.Wf} 
 of the Standard Library.

\index{hydras}{Library Hydra!Predicates!Termination}

\input{movies/snippets/Hydra_Definitions/TerminationDef}



Let $B$ be an instance of class \texttt{Battle}. A \emph{variant} for $B$ consists
in a well-founded relation $<$  on some type \texttt{A}, and a function
(also called a \emph{measure}) $m:\texttt{Hydra}\rightarrow A$ such that for any successive steps $(i,h)$ and $(1+i,h')$  of a battle in $B$, the inequality $m(h')<m(h)$ holds.


\vspace{4pt}
\noindent

\emph{From Module~\href{../theories/html/hydras.Hydra.Hydra_Definitions.html\#Hvariant}{Hydra.Hydra\_Definitions}}


\label{sect:hvariant-def}

\index{hydras}{Library Hydra!Type classes!Hvariant}

\input{movies/snippets/Hydra_Definitions/HvariantDef}

\index{hydras}{Exercises}

\begin{exercise}
 Prove that, if there exists some  instance of (\texttt{Hvariant Lt wf\_Lt $B$ $m$}), then there exists no infinite battle in  $B$.
\end{exercise}




\subsection{A  small proof of impossibility}
%\index{coq}{Proofs of impossibility}

\label{omega-case}

When one wants to prove a termination theorem with the help of a variant, 
one has to consider first a well-founded set $(A,<)$, then a strictly decreasing measure on this set.  The following  lemma shows that,  if  the order structure $(A,<)$ is too simple, it is useless to look for a convenient measure, which simply no exists. Such kind of result is useful, because it saves you time and effort.


The best known well-founded order is the natural order on the set $\mathbb{N}$ of natural numbers (the type \texttt{nat} of Standard library). It would be interesting to look for some measure $m:\texttt{nat}\arrow\texttt{nat}$ and prove it is a variant.

Unfortunately, we can prove that 
\emph{no} instance of class (\texttt{WfVariant round Peano.lt $m$}) can be built, where
$m$ is \emph{any} function of type \texttt{Hydra $\arrow$ nat}.


Let us present the main steps of that proof, the script of which  is in the module ~\href{../theories/html/hydras.Hydra.Omega_Small.html}{Hydra/Omega\_Small.v} \footnote{ The name of this file means ``the ordinal $\omega$ is too small for proving the termination of [free] hydra battles ''. In effect, the elements of $\omega$, considered as a set, are just the natural numbers (see next chapter for more details)}.

%\subsubsection{Preliminaries}


Let us assume there exists some variant $m$ from \texttt{Hydra} into (\texttt{nat},$<$)  for proving
    the  termination of all hydra battles.

\input{movies/snippets/Omega_Small/omegaSmalla}
    
We define an injection $\iota$ from the type \texttt{nat} into \texttt{Hydra}.
For any natural number $i$, $\iota(i)$ is the hydra composed of a foot and
$i+1$ heads at height $1$. For instance, Fig.~\ref{fig:flower} represents the hydra $\iota(3)$.

\begin{figure}[htb]
\centering
\begin{tikzpicture}[very thick, scale=0.5]
\node (foot) at (4,0) {$\bullet$};
\node (N1) at (2,2) {$\Smiley[2][green]$};
\node (N2) at (4,2) {$\Smiley[2][green]$};
\node (N3) at (6,2) {$\Smiley[2][green]$};
\node (N4) at (8,2) {$\Smiley[2][green]$};
\draw (foot) to [bend left =25] (N1);
\draw (foot) to [bend left =15] (N2);
\draw (foot) to [bend right =15] (N3);
\draw (foot) to [bend right =25] (N4);
\end{tikzpicture}
\caption{\label{fig:flower}
The hydra $\iota(3)$}
\end{figure}

\input{movies/snippets/Omega_Small/iotaDef}

Let us consider now some hydra \texttt{big\_h} out of the range of the injection $\iota$ (see Fig.~\vref{fig:h-omega-omega}).

\begin{figure}[htb]
\centering
\begin{tikzpicture}[very thick, scale=0.5]
\node (foot) at (2,0) {$\bullet$};
\node (N1) at (2,2) {$\bullet$};
\node (N2) at (2,4) {$\Smiley[2][green]$};
\draw (foot) -- (N1);
\draw (N1) to  (N2);
\end{tikzpicture}
\caption{\label{fig:h-omega-omega}}
 The hydra \texttt{big\_h}.
\end{figure}

\input{movies/snippets/Omega_Small/bigHDef}


 Using the functions $m$ and $\iota$, we define a second hydra \texttt{small\_h}, and show
 there is a one-round battle that transforms \texttt{big\_h} into \texttt{small\_h}. Please note that,
due to the hypothesis \texttt{Hvar}, we are interested in the termination of \emph{free} battles. 
There is no problem to consider a round with (\texttt{m big\_h}) as the replication factor.


\input{movies/snippets/Omega_Small/smallHDef}


 
But, by hypothesis, $m$ is a variant. Hence, we infer the following inequality.

\vspace{4pt}

\input{movies/snippets/Omega_Small/mLt}


In order to get a contradiction, it suffices to  prove the inequality
$m(\texttt{big\_h}) \leq m(\texttt{small\_h})$ i.e.,  $m(\texttt{big\_h}\leq m(\iota (m(\texttt{big\_h})))$.


\input{movies/snippets/Omega_Small/mGea}


Intuitively, it means that, from any hydra of the form (\texttt{iota $i$}), the battle will 
take (at least) $i$ rounds. Thus the associated measure cannot be less than $i$.
Technically, we prove this lemma by Peano induction on $i$.

\begin{itemize}
\item The base case $i=0$ is trivial
\item Otherwise, let $i$ be any natural number and assume  the inequality
  $i \leq m(\iota(i))$.
  \begin{enumerate}
  \item  But the hydra $\iota(S(i))$ can be transformed in one round into
    $\iota(i)$ (by losing its rightmost head, for instance)
  \item Since $m$ is a variant, we have $m(\iota(i)) < m(\iota(S(i)))$,
    hence  $i< m(\iota(S(i)))$, which implies  $S(i)\leq  m(\iota(S(i)))$.
  \end{enumerate}
\end{itemize}

We are now ready to complete our impossibility proof.

\vspace{4pt}

\inputsnippets{Omega_Small/mGeb, Omega_Small/mGez,
  Omega_Small/omegaSmallz}
 


\index{hydras}{Exercises}

\begin{exercise}
Prove that there exists no variant $m$ from \texttt{Hydra} into \texttt{nat} for proving
    the  termination of all \emph{standard} battles.
\end{exercise}






\subsubsection{Conclusion}

In order to build a variant for proving the termination of all hydra battles, we need to consider order structures more complex than the usual order on type \texttt{nat}. 
The notion of \emph{ordinal number} provides a catalogue of well-founded order types.
For a reasonably large bunch of ordinal numbers, \emph{ordinal notations} are data-types which allow the \coq{} user to define functions, to compute and prove some properties, for instance by reflection.

The next chapter is dedicated to a generic formalization of ordinal notations, and chapter~\ref{chap:T1} to a proof of termination of all hydra battles with the help of an ordinal notation for the interval $[0,\epsilon_0)$\,\footnote{We use the mathematical notation $[a,b)$ for the interval $\{x|a\leq x < b\}$.}.
\index{maths}{Notations!Interval}

\include{ordinal-chapter}
\include{epsilon0-chapter}
\include{ks-chapter}
\include{large-sets-chapter}
\include{gaia-chapter}
\chapter[Countable ordinals (after Sch\"{u}tte)]{Kurt Schütte's axiomatic definition of countable ordinals}

\label{chap:schutte} 
%ON

In the present chapter, we  compare our implementation of the segment $[0,\epsilon_0)$ with a mathematical text in order to ``validate'' our constructions.
Our reference here is the axiomatic definition of the set of countable ordinals,
in chapter V of Kurt Schütte's book `` Proof Theory ''~\cite{schutte}.

\begin{remark}
\emph{In all this chapter, the word ``ordinal'' will be considered as a synonymous of
``countable ordinal''}  
\end{remark}



Schütte's definition of countable ordinals relies on the following three axioms:

There  exists a strictly ordered set , such that
\begin{enumerate}
\item  $(\mathbb{O},<)$ is well-ordered
\item Every bounded subset of $\mathbb{O}$  is countable
\item Every countable subset of $\mathbb{O}$  is bounded.
\end{enumerate}

Starting with these three axioms, Schütte re-defines the vocabulary about ordinal numbers: the null ordinal $0$, limits and successors, the addition of ordinals, the infinite ordinals $\omega$, $\epsilon_0$, $\Gamma_0$, etc.

This chapter describes an adaptation to \coq{} of Schütte's axiomatization. 
 Unlike the rest of our libraries, our library
\href{../theories/html/hydras.Schutte.Schutte.html}{hydras.Schutte}
is not constructive, and relies on several axioms.

\begin{itemize}
\item First, please keep in mind  that the set of countable ordinals is not countable. Thus, we cannot hope to represent all countable ordinals as finite terms of an inductive type, which was possible with  the set of ordinals strictly less than $\epsilon_0$ (resp. $\Gamma_0$)
\item We tried to be as close as possible to K. Schütte's text, which uses ``classical'' mathematics : excluded middle, Hilbert's $\epsilon$ (choice) and Russel's $\iota$ (definite description) operators. Both operators allow us to write definitions close to the natural mathematical language, such as ``$\textrm{succ}$ is \emph{the} least ordinal strictly greater than $\alpha$''
\item Please note that only the library \href{../theories/html/hydras.Schutte.Schutte.html}{Schutte/*.v} is ``contaminated'' by axioms, and that the rest of our libraries remain constructive.
\end{itemize}

\section{Declarations and axioms}

Let us declare a type 
\texttt{Ord} for representing countable ordinals, and a binary relation
 \texttt{lt}. Note that, in our development, \texttt{Ord} is a type, while the \emph{set} of countable ordinals (called $\mathbb{O}$ by Schütte) 
is the full set over the type \texttt{Ord}.

\label{types:Ord} 

We use Florian Hatat's library on countable sets, written as he was a student of  \emph{\'Ecole Normale Supérieure de Lyon}. A set $A$ is countable if there is an injective function from $A$ to $\mathbb{N}$ (see 
Library \href{../theories/html/hydras.Schutte.Countable.html}%
{\texttt{Schutte.Countable}}).


\vspace{6pt}

\emph{From Module\href{../theories/html/hydras.Schutte.Schutte_basics.html}%
{\texttt{Schutte.Schutte\_basics}}}

\index{hydras}{Library Schutte!Types!Ord}

\input{movies/snippets/Schutte_basics/OrdDecl}


Schütte's first axiom tells that \texttt{lt} is a well order on the set 
\texttt{ordinal} (The  class \texttt{WO} is defined in
Module~\href{../theories/html/hydras.Schutte.Well_Orders.html}{Schutte.Well\_Orders.v}).

\index{hydras}{Library Schutte!Type classes!WO@ WO (well order)}

\label{types:WO}

\input{movies/snippets/Well_Orders/Mdecl}

\input{movies/snippets/Well_Orders/WODef}

\input{movies/snippets/Schutte_basics/AX1}

The second and third axioms say that a subset $X$ of $\mathbb{O}$ is
(strictly) bounded if and only if it is countable. 

\input{movies/snippets/Schutte_basics/AX23}


\texttt{AX2} and \texttt{AX3} could have been replaced by a single axiom (using the \texttt{iff} connector), but we decided to respect as most as possible the structure of Schütte's definitions.

\section{Additional  axioms}

The adaptation of Schütte's mathematical discourse to \coq{} led us to
import a few axioms from the standard library. We encourage the reader to consult \coq{}'s FAQ about the safe use of axioms
 \url{https://github.com/coq/coq/wiki/The-Logic-of-Coq#axioms}.

\subsubsection{Classical logic}

In order to work with classical logic, we import the module
\href{https://coq.inria.fr/distrib/current/stdlib/Coq.Logic.Classical.html}{Coq.Logic.Classical}  of \coq{}'s standard library, specifically the following axiom:

\begin{Coqsrc}
 Axiom classic : forall P:Prop, P \/ ~P.
\end{Coqsrc}


\subsubsection{Description operators}

In order to respect Schütte's style, we imported also the library 
\href{https://coq.inria.fr/distrib/current/stdlib/Coq.Logic.Epsilon.html}{\texttt{Coq.Logic.Epsilon}}.  The rest of this section presents a few examples of
how Hilbert's choice operator and Church's definite description allow us
 to write understandable definitions (close to the mathematical natural language).

\subsubsection{The definition of zero}

According to the  definition of a well order, every non-empty subset of \texttt{Ord} has a least element. Furthermore, this least element is unique. We would like to call this element  \texttt{zero}.

\input{movies/snippets/Schutte_basics/iotaDemoa}
\input{movies/snippets/Schutte_basics/iotaDemob}




Indeed, the basic logic of  \coq{} does not allow us to eliminate a proof of a proposition 
$\exists!\,x:A,\,P(x)$ for building a term whose type lies in the sort \texttt{Type}. 
The reasons for this impossibility are explained in many documents~\cite{BC04, chlipalacpdt2011, Coq}.

Let us import the library \texttt{Coq.Logic.Epsilon}, which contains the following axiom and lemmas.

\input{movies/snippets/MoreEpsilonIota/EpsilonStatement}


Hilbert's $\epsilon$ \emph{operator} is derived from this  axiom.

\input{movies/snippets/MoreEpsilonIota/EpsilonDef}


If we consider the \emph{unique existential} quantifier $\exists!$, we obtain
Church's \emph{definite description operator}.

\input{movies/snippets/MoreEpsilonIota/iotaDef}

Indeed, the operators \texttt{epsilon} and \texttt{iota} allowed us to make our definitions 
quite close to Schütte's text. Our libraries \href{../theories/html/hydras.Schutte.MoreEpsilonIota.html}%
{\texttt{Schutte.MoreEpsilonIota}}
and
\href{../theories/html/hydras.Schutte.PartialFun.html}%
{\texttt{Schutte.PartialFun}} are extensions of \texttt{Coq.logic.Epsilon} for making easier 
such definitions. See also an article in french~\cite{PCiota}. 


\input{movies/snippets/MoreEpsilonIota/Defs}


In order to use these tools,  we have to tell \coq{}  that the declared type \texttt{Ord} is not empty:

\input{movies/snippets/Schutte_basics/inhOrd}




We are now able to define \texttt{zero} as the least ordinal. For this purpose,
we define a function returning the least element of any [non-empty]  subset.


\emph{From Module\href{../theories/html/hydras.Schutte.Well_Orders.html}%
  {\texttt{Schutte.Well\_Orders}}}

\input{movies/snippets/Well_Orders/MDecl}
\input{movies/snippets/Well_Orders/theLeast}



\vspace{4pt}

From Module \href{../theories/html/hydras.Schutte.Schutte_basics.html}%
{\texttt{~Schutte.Schutte\_basics}}

\label{Constants:zero:Ord}
\index{hydras}{Library Schutte!Constants!zero}

\input{movies/snippets/Schutte_basics/zeroDef}

We want to prove now that zero is less than or equal to any ordinal number.

\input{movies/snippets/Schutte_basics/zeroLe}


\subsubsection{Remarks on \texttt{epsilon} and \texttt{iota}}

 What would happen in case of a misuse of \texttt{epsilon} or \texttt{iota} ?
For instance, one could give a unsatisfiable specification to \texttt{epsilon} or 
a specification for \texttt{iota} that admits several realizations.

Let us consider an example:

\input{movies/snippets/Schutte_basics/BadBottoma}

Since we won't be able to prove the proposition
\linebreak \Verb|(exists! a: Ord, least_member (Empty_set Ord) a)|, the only properties we would be able to prove about \texttt{bottom} would be \emph{trivial} properties, 
\emph{i.e.}, satisfied by \emph{any} element of type \texttt{Ord}, like for instance
\texttt{bottom = bottom}, or \texttt{zero <= bottom}.

\input{movies/snippets/Schutte_basics/trivialProps}

On the other hand, the following attempt fails, because of the unprovable first subgoal (please notice that the second subgoal is easy to solve !).

\input{movies/snippets/Schutte_basics/Failure}


In short, using \texttt{epsilon} and \texttt{iota} in our implementation of countable ordinals after Schütte has two main advantages.


\begin{itemize}
\item It allows us to give a \emph{name} (using \texttt{Definition}) to witnesses 
of existential quantifiers (let us recall that, in classical logic, one may consider non-constructive proofs of existential statements)
\item By separating definitions from proofs of [unique] existence, one may make definitions  more concise and readable. Look for instance at 
the definitions of  \texttt{zero}, \texttt{succ}, \texttt{plus}, etc. in the rest of this chapter.
\end{itemize}
%%%% ICI ICI 

\section{The  successor function}

The definition of the function \texttt{succ:Ord -> Ord} is very concise. The successor of any ordinal $\alpha$ is the smallest ordinal strictly greater than $\alpha$.

\label{Functions:succ-sch}
\index{hydras}{Library Schutte!Functions!succ}

\input{movies/snippets/Schutte_basics/succDef}

Using \texttt{succ}, we define the following predicates.

\input{movies/snippets/Schutte_basics/isSuccIsLimit}




% \begin{remark}
% Please look at remark~\vref{warning:coercions}.

How do we prove properties of the successor function?
First, we make its specification explicit.

\input{movies/snippets/Schutte_basics/succSpec}

Then, we prove that our function \texttt{succ} meets this specification. 

\input{movies/snippets/Schutte_basics/succOka}


We have now to prove that the set of all ordinals strictly greater than $\alpha$ has a unique least element. But the singleton set $\{\alpha\}$ is countable, hence  bounded (by the axiom \texttt{AX3}). Hence; the set $\{\beta\in\mathbb{O}|\alpha < \beta\}$ is not empty
and therefore has a unique least element.

The rest of the \coq{} proof script is quite short.

\input{movies/snippets/Schutte_basics/succOkb}

We can ``uncap'' the description operator for proving properties of the
\texttt{succ} function.

\input{movies/snippets/Schutte_basics/succProps}



\section{Finite ordinals}

Using \texttt{succ}, it is now easy to define recursively all the finite ordinals.

\label{sect:notation-F-sch}

\input{movies/snippets/Schutte_basics/finiteDef}

\section{The definition of \texttt{omega}}
In order to define $\omega$, the first infinite ordinal, we use an operator which
``returns'' the least upper bound (if it exists) of a subset $X\subseteq \mathbb{O}$.
For that purpose, we first use a predicate:
(\texttt{is\_lub $D$ \textit{lt} $X$ $a$}) if $a$ belongs to $D$ and is the least 
upper bound  of $X$ (with respect to \textit{lt}).


\input{movies/snippets/Lub/isLubDef}

\input{movies/snippets/Schutte_basics/supDef}


Then, we define the function \texttt{omega\_limit} which returns the least upper bound 
of the  (denumerable) range of any sequence \texttt{s: nat -> Ord}. 
By \texttt{AX3} this range is bounded, hence the set of its upper bounds is not empty and has a least element.
Then we define \texttt{omega} as the limit of the sequence of finite ordinals.

\input{movies/snippets/Schutte_basics/omegaDef}

\label{sect:notation-omega}




Among the numerous properties of the ordinal $\omega$, let us quote the following ones
(proved in Module 
\href{../theories/html/hydras.Schutte.Schutte_basics.html\#finite_lt_omega}{\texttt{Schutte.Schutte\_basics}})

\input{movies/snippets/Schutte_basics/omegaPropsa}
\input{movies/snippets/Schutte_basics/omegaPropsb}
\input{movies/snippets/Schutte_basics/omegaPropsc}
\input{movies/snippets/Schutte_basics/omegaPropsd}



\subsection{Ordering functions and ordinal addition}

After having defined the finite ordinals and the infinite ordinal $\omega$, we  define the sum $\alpha+\beta$ of two countable ordinals.
Schütte's definition looks like the following one:

\begin{quote}
``$\alpha+\beta$ is the $\beta$-th ordinal greater than or equal to $\alpha$''
\end{quote}


The purpose of this section is to give a meaning to the construction
``the $\alpha$-th element of $X$''  where $X$ is any non-empty subset of $\mathbb{O}$.
We follow Schütte's approach, by defining the notion of \emph{ordering functions},
a way to associate a unique ordinal to each element of $X$.
Complete definitions and proofs can be found in Module
 \href{../theories/html/hydras.Schutte.Ordering_Functions.html}%
{\texttt{Schutte.Ordering\_Functions}} ).

\subsection{Definitions}

A \emph{segment} is a set $A$ of ordinals such that, whenever  $\alpha\in A$ and
$\beta<\alpha$, then $\beta\in A$; a segment is  \emph{proper} if it strictly included in $\mathbb{O}$.

\input{movies/snippets/Ordering_Functions/segmentDef}


Let  $A$ be a segment, and $B$ a subset of $\mathbb{O}$ : an \emph{ordering function for $A$ and  $B$} is a strictly increasing bijection from $A$ to $B$.
The set $B$ is said to be an \emph{ordering segment} of $A$.
Our definition in \coq{} is a direct translation of the mathematical text of~\cite{schutte}.

\index{maths}{Ordinal numbers!Ordering functions}
\index{hydras}{Library Schutte!Predicates!ordering function@ordering\_function}

\input{movies/snippets/Ordering_Functions/orderingFunctionDef}

We are now able to associate with any subset $B$ of $\mathbb{O}$ its ordering segment and ordering function.

\input{movies/snippets/Ordering_Functions/ordDef}


Thus (\texttt{ord $B \;\alpha$}) is the $\alpha$-th element of $B$.
Please note that the last definition uses the epsilon-based operator \texttt{some} and
not \texttt{the}. This is due to the fact that we cannot prove the unicity (w.r.t. Leibniz' equality) of the ordering function of a given set. 
By contrast, we admit the axiom  \texttt{Extensionality\_Ensembles}, from the library 
\href{https://coq.inria.fr/distrib/current/stdlib/Coq.Sets.Ensembles.html}{Coq.Sets.Ensembles}, so we use the operator \texttt{the} in the definition of
\texttt{the\_ordering\_segment}.

One of the main theorems of
\href{../theories/html/hydras.Schutte.Ordering_Functions.html\#ordering_function_ex}%
{\texttt{Ordering\_Functions}} 
associates a unique segment and a unique (up to extensionality) ordering function to every subset $B$ of $\mathbb{O}$.

\input{movies/snippets/Ordering_Functions/orderingFunctionEx}

Thus,  our function \texttt{ord}  which enumerates the elements of $B$ is defined in a non-ambiguous way.
Let us quote the following theorems (see Library \linebreak
\href{../theories/html/hydras.Schutte.Ordering_Functions.html}%
{\texttt{Schutte.Ordering\_Functions}} for more details).
 
\input{movies/snippets/Ordering_Functions/orderingLe}
\input{movies/snippets/Ordering_Functions/Th1352}


\subsection{Ordinal addition}

We are now ready to define and study addition on the type \texttt{Ord}.
The following definitions and proofs can be consulted in Module
\href{../theories/html/hydras.Schutte.Addition.html}%
{\texttt{Schutte.Addition.v}}.

\index{hydras}{Library Schutte!Functions!plus}

\input{movies/snippets/Addition/additionDef}

In other words,  $\alpha + \beta$ is the  $\beta$-th ordinal greater than or equal to $\alpha$. 
Thanks to generic properties of ordering functions, we can show the following 
properties of addition on $\mathbb{O}$. First, we prove a useful lemma:

\input{movies/snippets/Addition/plusElim}

As a use-case, let us prove that $0$ is a right neutral element of $+$.

\input{movies/snippets/Addition/alphaPlusZero}


The following lemmas are proved the same way.

\input{movies/snippets/Addition/bunchOfLemmas}


The following lemmas are not direct applications of \texttt{plus\_elim}.

\input{movies/snippets/Addition/plusAssoc}
\input{movies/snippets/Addition/finitePlusInfinite}
  


It is interesting to compare the proof of these lemmas with the
computational proofs of the corresponding statements in Module
\href{../theories/html/hydras.Epsilon0.T1.html}%
{\texttt{Epsilon0.T1}}. 
For instance, the proof of the lemma 
\texttt{one\_plus\_omega} uses the continuity of ordering functions (applied to  \texttt{(plus 1)}) and compares the limit of the $\omega$-sequences $i_{(i \in \mathbb{N})}$ and
$(1+i)i_{(i \in \mathbb{N})}$, whereas in the library  \texttt{Epsilon0/T1}, the equality 
$1+\omega=\omega$ is just proved with \texttt{reflexivity}!



\subsubsection{Multiplication by a natural number}

The multiplication of an ordinal by a natural number is defined in terms of addition.
This operation is useful for the study of Cantor normal forms.

\input{movies/snippets/Addition/multFin}

\section{The exponential of basis \texorpdfstring{$\omega$}{omega}}

In this section, we define the function which maps any $\alpha\in\mathbb{O}$ to
the ordinal  $\omega^\alpha$, also written 
$\phi_0(\alpha)$. 
It is an opportunity to apply the definitions and results of the preceding section. 
Indeed,  Schütte first defines a subset of $\mathbb{O}$: the set of additive principal ordinals, and $\phi_0$  is just defined as the ordering function of this set.

\subsection{Additive principal ordinals}

\index{maths}{Ordinal numbers!Additive principal ordinals}
\index{hydras}{Library Schutte!Predicates!AP@AP (additive principal ordinals)}

\begin{definition}
A non-zero ordinal  $\alpha$ is said to be \emph{additive principal} if, for all  $\beta<\alpha$, $\beta+\alpha$ is equal to  $\alpha$.
We call \texttt{AP} the set of additive principal ordinals.

\end{definition}



\noindent\emph{From Module \href{../theories/html/hydras.Schutte.AP.html}%
{\texttt{Schutte.AP}}}

\input{movies/snippets/AP/APDef}

\subsection{The function \texttt{phi0}}

Let us call  $\phi_0$ the ordering function of \texttt{AP}.
In the mathematical text, we shall use indifferently the notations  $\omega^\alpha$ and $\phi_0(\alpha)$. 

\index{hydras}{Library Schutte!Functions!phi0}

\input{movies/snippets/AP/phi0Def}


\subsection{Omega-towers and the ordinal \texorpdfstring{$\epsilon_0$}{epsilon0}}


Using $\phi_0$, we can define recursively the set of finite omega-towers.

\input{movies/snippets/AP/omegaTower}



\label{sect:epsilon0-as-limit}
Then, the ordinal  $\epsilon_0$ is defined as the limit of the sequence of all finite towers (a kind of infinite tower).

\input{movies/snippets/AP/epsilon0Def}


The rest of our library \texttt{AP} is devoted to the proof of properties of additive principal ordinals, hence of the ordering function  $\phi0$ and the ordinal $\epsilon_0$ (which we could not express within the type \texttt{T1}).

\subsection{Properties of the set  \texttt{AP}}

The set of additive principal ordinals is not empty: it contains at least the ordinals  $1$ and  $\omega$. 

\inputsnippets{AP/APOne,AP/APOne_least_AP,AP/APOne_AP_omega,AP/APOne_omega_second_AP}

The set  \texttt{AP} is  \emph{closed} under addition, and unbounded.
\label{lemma:AP-plus-closed}

\input{movies/snippets/AP/APPlusClosed}

\input{movies/snippets/AP/APUnbounded}

Finally, \texttt{AP} is (topologically) \emph{closed} and ordered by the segment of all countable ordinals.

\index{hydras}{Library Schutte!Predicates!Closed}

From Module \href{../theories/html/hydras.Schutte.Schutte_basics.html}%
{\texttt{~Schutte.Schutte\_basics}}


\input{movies/snippets/Schutte_basics/ClosedDef}
\input{movies/snippets/AP/APClosed}


\subsubsection{Properties of the function \texorpdfstring{$\phi_0$}{phi0}}
 
The ordering function $\phi_0$ of the set \texttt{AP} is defined on the full set $\mathbb{O}$ and is continuous (Schütte calls such a  function  \emph{normal}).

\begin{Coqsrc}
Theorem normal_phi0 : normal phi0 AP.
\end{Coqsrc}

The following properties come from  the definition of $\phi_0$ as the ordering function of \texttt{AP}. It may be interesting to compare these proofs with the computational ones described in Chapter ~\ref{chap:T1}.

\input{movies/snippets/AP/APPhi0}

\subsection{A last example}
\label{Ex42-schutte}

Let us prove again the equality $\omega+42+\omega^2= \omega^2$.Let us recall that $\omega^2$ is an abbreviation of $\phi_0(2)$,
\emph{i.e} the third  additive principal ordinal.

\inputsnippets{Schutte/Ex42a}


Our proof is very different from the computational proof of Sect~\vref{Ex42-E0}.
By definition of additive principal ordinals, 
it suffices to prove the inequality $\omega+42< \phi_0(2)$.

\inputsnippets{Schutte/Ex42b}

Since the set \texttt{AP} of additive principals  is closed under addition
(by Lemma \texttt{AP\_plus\_closed}, page~\pageref{lemma:AP-plus-closed}) , it suffices to prove the inequalities $\omega<\phi_0(2)$ and $42<\phi_0(2)$.

\inputsnippets{Schutte/Ex42d, Schutte/Ex42c, Schutte/Ex42e}

\section{More about \texorpdfstring{$\epsilon_0$}{\texttt{epsilon0}}}

Let us recall that the limit ordinal  $\epsilon_0$ cannot be written within the type \texttt{T1}. Since we are now considering the set of all countable ordinals, we can now prove some properties of this ordinal.


We prove the inequality  $\alpha<\omega^\alpha$ whenever $\alpha < \epsilon_0$.
\emph{Note that this condition was implicit in Module
\href{../theories/html/hydras.Epsilon0.T1.html\#lt_phi0}{Epsilon0.T1}.}

\input{movies/snippets/AP/ltPhi0}


The proof is as follows:
\begin{enumerate}
\item Since $\alpha<\epsilon_0$, consider the least $i$ such that $\alpha$ is strictly less than the omega-tower of height $i$.
\item
  \begin{itemize}
  \item If $i=0$, then the result is trivial (because $\alpha=0$)
 \item  Otherwise let $i=j+1$; 
          $\alpha$ is greater than or equal to the omega-tower of height $j$.
         By monotony,  $\phi_0(\alpha)$ is greater than or equal to 
        the omega-tower of height $j+1$, thus strictly greater than $\alpha$
  \end{itemize}
 \end{enumerate}

Moreover,  $\epsilon_0$ is the least ordinal $\alpha$ that verifies the equality 
$\alpha = \omega^\alpha$, in other words, the least fixpoint of the function  $\phi_0$.

\input{movies/snippets/AP/epsilon0Lfp}

\section{Critical ordinals}

\index{maths}{Ordinal numbers!Critical ordinals}
\index{hydras}{Library Schutte!Predicates!Cr@Cr (critical ordinals)}

For any  (countable) ordinal $\alpha$, the set $\textit{Cr}(\alpha)$ is inductively defined 
as follows by Schütte (p.81 of~\cite{schutte}).

\begin{quote}
  \begin{itemize}
  \item $\textit{Cr}(0)$ is the set \textit{AP} of additive principal ordinals.
  \item If $0<\alpha$, then $\textit{Cr}(\alpha)$ is the intersection of all the sets of fixpoints of the $\textit{Cr}(\beta)$ for $\beta<\alpha$.
  \end{itemize}
\end{quote}

This definition is translated in \coq{} in 
Module \href{../theories/html/hydras.Schutte.Critical.html}%
{\texttt{Schutte.Critical}}, as the least fixpoint of a functional. 

\input{movies/snippets/Critical/CrDef}

Lets us denote by $\phi_\alpha$ the ordering function of the set $\textit{Cr}(\alpha)$ and by $A_\alpha$ its ordering segment.

\input{movies/snippets/Critical/phiDef}

\label{sect:phi-schutte}



For instance,  we prove that $\textit{Cr}(0)$ is the set of additive principals and that $\epsilon_0$
belongs to $\textit{Cr}(1)$.


\inputsnippets{Critical/CrZeroAP,
  Critical/epsilon0Cr1}


\index{hydras}{Exercises}

\begin{exercise}
 Prove that $\epsilon_0$ is the least element of $\textit{Cr}(1)$.
\end{exercise}


\subsection{A flavor of infinity}



The family of the $\textit{Cr}(\alpha)$s is made of infinitely many unbounded (hence infinite) sets.
Let us quote Lemma 5, p. 82  of~\cite{schutte}:
\begin{quote}
  For all $\alpha$, the set $\textit{Cr}(\alpha)$ is closed (for the least upper bound of non-empty countable sets) and unbounded.
\end{quote}

We prove this result by a transfinite induction on $\alpha$ of the conjunction of  both properties.




\index{maths}{Transfinite induction}

\input{movies/snippets/Critical/Lemma5}



\section{Cantor normal form}

The notion of Cantor normal form is defined for all countable ordinals.
Nevertheless, note that, contrary to the implementation based on type \texttt{T1},
the Cantor normal form of an ordinal $\alpha$ may contain $\alpha$ as a 
sub-term\footnote{This would prevent us from trying to represent Cantor normal forms as finite trees (like in Sect.~\ref{sec:T1-inductive-def})}.

\index{maths}{Ordinal numbers!Cantor normal form}
\index{hydras}{Library Schutte!Predicates!is\_cnf\_of@is\_cnf\_of (to be a Cantor normal form of}

We represent  Cantor normal forms as lists of ordinals.
A  list $l$ is a Cantor normal form of a given ordinal $\alpha$ if it satisfies two conditions:



\begin{itemize}
\item The list  $l$ is sorted (in decreasing order) w.r.t. the order $\leq$
\item The sum of all the  $\omega^{\beta_i}$ where the $\beta_i$ are the terms of $l$ (in this order) is equal to $\alpha$.
\end{itemize}



\vspace{4pt}

\noindent\emph{From \href{../theories/html/hydras.Schutte.CNF.html\#cnf_t}%
{\texttt{Schutte.CNF}}}

\input{movies/snippets/CNF/Defs}



\index{maths}{Transfinite induction}

By transfinite induction on $\alpha$, we prove that every countable ordinal $\alpha$ 
 has at least a Cantor normal form.

\input{movies/snippets/CNF/cnfExists}

By structural induction on lists, we prove that this normal form is unique.


\inputsnippets{CNF/cnfUnicity,CNF/cnfExUnique}



Finally, the following two lemmas relate  $\epsilon_0$ with Cantor normal forms.

If $\alpha<\epsilon_0$, then the Cantor normal form of $\alpha$ is made of ordinals strictly less than $\alpha$.

\input{movies/snippets/CNF/cnfLtEpsilon0}



\index{hydras}{Exercises}

\begin{exercise}
Please consider the following statement :

\begin{Coqsrc}
Lemma cnf_lt_epsilon0_iff : 
 forall l alpha, 
   is_cnf_of alpha l ->  
   (alpha < epsilon0 <->  Forall (fun beta =>  beta < alpha) l).
\end{Coqsrc}

Is it true ?

\emph{You may start this exercise with the file
    \href{https://github.com/coq-community/hydra-battles/tree/master/exercises/ordinals/schutte_cnf_counter_example.v}{exercises/ordinals/schutte\_cnf\_counter\_example.v}.}
\end{exercise}

Finally, the Cantor normal form of $\epsilon_0$ is just $\omega^{\epsilon_0}$.

\input{movies/snippets/CNF/cnfOfEpsilon0}

\index{hydras}{Projects}

\begin{project}
Implement pages 82 to 85 of~\cite{schutte} (critical, strongly critical, maximal critical ordinals, Feferman's ordinal $\Gamma_0$).
\end{project}

\begin{remark}
The sub-directory
    \href{https://github.com/coq-community/hydra-battles/tree/master/theories/ordinals/Gamma0}{theories/oridinals/Gamma0} contains an (incomplete, still undocumented) implementation of the set of ordinals below $\Gamma_0$, represented in Veblen normal form. 
\end{remark}

\section{An embedding of \texttt{T1} into \texttt{Ord}}


Our library 
\href{../theories/html/hydras.Schutte.Correctness_E0.html}%
{\texttt{Schutte.Correctness\_E0}} establishes the link between two very different modelizations of ordinal numbers. In other words, it ``validates'' a data structure in terms of
a classical mathematical discourse considered as a model. 
First, we define a function from \texttt{T1} into  \texttt{Ord} by structural recursion.

\input{movies/snippets/Correctness_E0/injectDef}


This function enjoys good commutation properties with respect to the main operations which
allow us to build Cantor normal forms.

\input{movies/snippets/Correctness_E0/commutationLemmas}
\input{movies/snippets/Correctness_E0/injectPlus}
\input{movies/snippets/Correctness_E0/injectMultFinR}

% \begin{Coqsrc}
% Theorem inject_mono (beta gamma : T1) :
%   T1.lt  beta gamma -> 
%   T1.nf beta -> T1.nf gamma -> 
%   inject beta < inject gamma.

% Theorem inject_injective (beta gamma : T1) : nf beta -> nf gamma ->
%   inject beta = inject gamma -> beta = gamma.
% \end{Coqsrc}

Finally, we prove that \texttt{inject} is a bijection from the set of all terms of \texttt{T1} in normal form to the set 
\texttt{members epsilon0} of the elements of \texttt{Ord} strictly less than  $\epsilon_0$.

\input{movies/snippets/Correctness_E0/injectLtEpsilon0}
\input{movies/snippets/Correctness_E0/embedding}



\subsection{Remarks}
Let us recall that the library \href{../theories/html/hydras.Schutte.Schutte.html}%
{\texttt{Schutte}} depends on five \emph{axioms} and lies explicitly in the  
framework of classical logic with a weak version of the axiom of choice
(please look at the documentation of
\href{https://coq.inria.fr/distrib/current/stdlib/Coq.Logic.ChoiceFacts.html}{\texttt{Coq.Logic.ChoiceFacts}}).
Nevertheless, the other modules:
\href{../theories/html/hydras.Epsilon0.Epsilon0.html}%
{\texttt{Epsilon0}},
\href{../theories/html/hydras.Hydra.Hydra.html}%
{\texttt{Hydra}}, et 
\href{../theories/html/hydras.Gamma0.Gamma0.html}%
{\texttt{Gamma0}}
do not import any axioms and are really constructive.

\index{hydras}{Projects}
\begin{project}
There is no construction of ordinal multiplication in~\cite{schutte}. 
It would be interesting to derive this operation from Schütte's axioms,
and prove its consistence with multiplication in ordinal notations for 
$\epsilon_0$ and $\Gamma_0$.
\end{project}

\section{Related work}

In~\cite{grimm:hal-00911710}, José Grimm establishes the consistency between our ordinal notations \texttt{T1} and \texttt{T2} (Veblen normal form) and his implementation
of ordinal numbers after Bourbaki's set theory.

The Gaia project ~\url{https://github.com/coq-community/gaia} maintains Grimm's  theory of ordinals as part of coq-community on GitHub. Integration
of the present ordinal theory with Gaia, i.e., relating the different notions of ordinals
and transferring relevant results, is an interesting project.
First experiments in that direction are developped in
the \href{https://github.com/coq-community/hydra-battles/blob/master/theories/gaia/}{theories/gaia/} directory.

\include{Gamma0-chapter}


\part{Ackermann, G\"{o}del, Peano\,\dots}
This part contains comments, example  and exercises about Russel O'Connor's work on G\"{o}del's first incompleteness theorem.  This work is maintained by Coq community~\cite{CoqCommunity} volunteers: (projects Goedel~\cite{Goedel} and Hydra-battles~\cite{HydraBattles}).
The main reference to this work is  Russel O'Connor's article~\cite{OConnor05}, which we strongly encourage the reader to 
consult regularly.



Some changes are made to the aforementionned libraries, mainly because of the recent evolution of coq and its libraries. Nevertheless, the definitions, lemmas and theorems of the original contribution have been kept in this new release.

\emph{\color{red!80}These maintenance and documentation 
jobs have just started, and will probably be long to complete. Help is welcome!}

\section{File contents}

All Russel O'Connor's files have been stored in two directories, in order to simplify packages maintenance. 

\begin{itemize}
\item \texttt{theories/goedel/}: Proofs which depend on 
\texttt{CoqPrime} package.
\item  \texttt{theories/ordinals/Ackermann/} : all the rest:
 definition of primitive recursive functions, first-order  logic,
Peano Arithmetics, G\"{o}del's encoding.
\item Some additions we made: examples, exercices, new notations, etc.,  are stored in a specific directory \texttt{theories/ordinal/MoreAck/}.
\end{itemize}


\begin{todo}
Add information on recent developments on formal 
proofs of G\"odel  incompleteness theorems. Justify the decision of working on \emph{this} development.  
\end{todo}

\section{Warning}
Russel O'Connors contribution contains more than 42 KLoc.
Since its construction, \coq{},  its libraries and recommended style have evolved a lot. We have just started to ``modernize'' this code. We apologize for provisional inconsistencies of presentation (code and documentation).

\include{chapter-primrec} 
\chapter{First Order Logic (in construction)}
\label{chap:fol}

\section{Introduction}

This chapter is devoted to the presentation of  data structures for representing terms and first order formulas over a ranked alphabet, and the basic functions and predicates over these types, more precisely:

\begin{itemize}
\item Abstract syntax of terms and formulas over a ranked alphabet composed of function and relation symbols.
\item Induction principles over terms and formulas.
\item Definition and main properties of substitution of terms to variables.
\end{itemize}


Although all the following constructions come directly from~\cite{Goedel}, we introduced minor changes (mainly syntax) to
take into account recent changes in \coq(new constructions, tactics, notations, etc.).


\section{Data types}

\subsection{Languages}

A \emph{language} is a structure composed of relation and function symbols, each symbol is given an \emph{arity} (number of arguments)\,\footnote{As suggested by Russel O'Connor in~\cite{OConnor05}, we consider two arity functions instead of a single function defined on the sum type \texttt{Relations + Functions}.}.

From~\href{../theories/html/hydras.Ackermann.fol.html}{Ackermann.fol}
\inputsnippets{fol/LanguageDef} 


\subsubsection{Example: $L$, a toy language}
In order to show a few simple examples of statements and proofs, we define a small language with very few symbols:
two constant symbols: $a$ and $b$, three function symbols $f$, $g$ and $h$  (of respective arity 1, 1 and 2), three propositional symbols $A$, $B$ and $C$, a one-place predicate symbol $P$, and a binary relational symbol $R$. 
 
From~\href{../theories/html/hydras.MoreAck.FolExamples.html}{MoreAck.FolExamples}.

\inputsnippets{FolExamples/ToyDef}

\begin{remark}
  \label{rem:underscores}
  The constructors of types \texttt{Rel} and \texttt{Fun} are suffixed by an underscore, in order to reserve the names \texttt{a}, \texttt{f}, \texttt{h}, \texttt{R}, etc. to the functions which build terms and formulas (please look at Sect~\ref{sec:fol-term-notations} and \ref{sec:fol-atomic-notations}).
\end{remark}


\subsection{Terms}

Given a language $L$, we define the type of \emph{terms} and
$n$-\emph{tuples of terms} over $L$.

\inputsnippets{fol/TermDef} 

\begin{remark}
This representation of terms uses mutually inductive data-types instead of lists or vectors of terms. Please see also Remark~\ref{hydra:mutually-inductive-vs-lists}.
\end{remark}


\begin{remark}[Variables]
In O'Connor's formalization of first-order logic,  variables are 
just natural numbers, and the \emph{coercion} from
\texttt{nat} to \texttt{Term L} is the constructor (\texttt{@var $L$}).
Although other choices may be considered : PHOAS, de Bruijn indices, etc,  we still the data structures of~\cite{Goedel}, in order not to break long proof scripts which use this representation (please look at Section 2 of~\cite{OConnor05} for a related discussion).
  
\end{remark}




\subsubsection{Examples}
\label{sect:folTermExamples}
\label{sec:fol-term-notations}

Let us build a few \gallina terms over our toy language, 
respectively corresponding to the terms $a$,
$f(a)$, $h(f(a),a)$, and $h(f(v_0),g(v_1))$.

First, in order to make terms on $L$ more readable, we introduce a few abbreviations.


From~\href{../theories/html/hydras.MoreAck.FolExamples.html}{MoreAck.FolExamples}.

\inputsnippets{FolExamples/termAbbreviations}

\inputsnippets{FolExamples/TermExamples1} 


The following ``term'' \texttt{t4} is not well formed, since the arity of $h$ is not respected\,\footnote{Strictly speaking, it's not a (well typed) term!}.

\inputsnippets{FolExamples/TermExamplesFail} 

\subsubsection{Other Languages}

\begin{todo}
Link to the chapter which presents \texttt{LNT} and \texttt{LNN}.
\end{todo}

\subsection{First-order formulas}



The type of first order formulas over $L$ is defined 
in~\href{../theories/html/hydras.Ackermann.fol.html}{Ackermann.fol} as an inductive data type, with a limited set of basic constructions:
\emph{term equalities} $t_1=t_2$,
\emph{atomic propositions} $R\;t_1\;\dots\;t_n$, where $R$ is a relation symbol of arity $n$,
\emph{implications} $A \rightarrow B$,
\emph{negations} $\sim\,A$,
and \emph{universal quantifications} $\forall\,i,\;A$.

\vspace{6pt}

\noindent From~\href{../theories/html/hydras.Ackermann.fol.html}{Ackermann.fol}

\inputsnippets{fol/FormulaDef}



\begin{remark}

In~\cite{Goedel}, no \emph{constructors} of type \texttt{(Formula L)} are associated with
disjunction, conjonction, logical equivalence and existential quantifier. These constructs are formalized through \emph{definitions}\footnote{Please keep in mind that we are considering a classical logic.}.

\inputsnippets{fol/FolFull, fol/folPlus}

This convention allows the user to reduce to 5 (instead of 10) the number of cases in ''\texttt{match F with \dots}'' constructs. On the other hand, some computation may expand a connective like $\vee$ or
$\wedge$, or an existential quantification into a ``basic'' formula (see Sect.\vref{sect:fol-issue}).
\end{remark}

\subsection{Examples}




Let us give a few examples of first-order formulas over $L$.
\label{fol:examplesf1f2f3}
\begin{description}
\item[F1] $R\;a\;b$
\item[F2] $\forall v_0\; v_1, R\;v_0\;v_1 \arrow R\;v_1\;v_0$
\item[F3] $ \forall v_0, v_0=a \vee \exists\;v_1,\, v_0= f(v_1)$
\item[F4] $(\forall v_1, v_0 = v_1) \vee \exists\,v_0\,v_1, v_0 \not= v_1$
\item[F5] $v_0 =a \vee v_0 = f(v_1)$
 \item[F6] $\forall v_0,\,\exists v_1, v_0= f(v_1)\wedge v_0\not= v_1$
\end{description}

Let us now define these formulas as terms of type
\texttt{(Formula L)}.

First, we define abbreviations for atomic formulas over $L$.

\noindent From~\href{../theories/html/hydras.MoreAck.FolExamples.html}{MoreAck.FolExamples}

\inputsnippets{FolExamples/toyNotationForm}

\label{sec:fol-atomic-notations}
\label{sect:folFormExamples}

\inputsnippets{FolExamples/FormExamples}


% The following interactions show how the formulas 
% of~\ref{sect:folFormExamples} may be 
% parsed or printed. 

% \noindent From~\href{../theories/html/hydras.MoreAck.FolExamples.html}{MoreAck.FolExamples}

% \vspace{6pt}

% \inputsnippets{FolExamples/toyNotationForm2}








\subsubsection{Bound variables}

In~\cite{Goedel}, there is no De Bruijn encoding of bound variables (see also~\cite{OConnor05}).

For instance, the term \texttt{(var 0)}
occurs freely and also inside the scope of a quantifier in the
formula \texttt{F4} \vpageref[above]{sect:folFormExamples}.

The following example shows two formulas which share the same structure, are logically equivalent, but are not Leibniz equal.



\vspace{4pt}

\noindent From~\href{../theories/html/hydras.MoreAck.FolExamples.html}{MoreAck.FolExamples}
  
\inputsnippets{FolExamples/boundVars}  


  \begin{todo}
   Link to the lemmas which attest the equivalence of these formulas (properties of substitution, logical equivalence).
  \end{todo}


  \index{ackermann}{Projects}
  \begin{project}
 Use a HOAS representation for FOL terms and formulas (without breaking proof scripts).   
  \end{project}



\section{A notation scope for first-order terms and formulas}
\label{sect:fol-notations}


 We use \coq's \texttt{Notation} features to print and parse terms and formulas  in a more readable form.
To this purpose, we build \texttt{fol\_scope}, a  notation scope
where the main connectives and quantifiers get a syntax close to \coq's.  Additionnally, a term of the form \texttt{(@var \_ $i$)} is
just printed and parsed  \texttt{v\#$i$}.

\inputsnippets{fol/folScope1}


The \texttt{\%fol} delimiter 
allows the user to distinguish \texttt{FOL} connectives from their \coq equivalent.
\emph{We discourage the reader from \emph{opening} \texttt{fol\_scope} and similar scopes : \texttt{nn\_scope}, \texttt{nt\_scope}, which would make disappear the \texttt{\%fol} suffix from the first-order formulas}.


\vspace{6pt}

\noindent From \href{../theories/html/hydras.Ackermann.fol.html}{Ackermann.fol}

\inputsnippets{FolExamples/toyNotationForm2}


\subsection{The issue with derived constructions}
\label{sect:fol-issue}


The connectives and quantifiers $\vee$, $\wedge$, $\exists$, etc. may raise an issue when printing computed formulas.
For instance, a formula like $F \wedge G$ could be transformed into $\sim(\sim F \vee \sim G)$,  and even
into $\sim(\sim\sim F \arrow \sim B)$, which would 
cause serious problems of readability.

In such a case, we propose to print such a formula as $F \wedge' G$, to make it syntactically distinct but very similar to $F \wedge G$.

\inputsnippets{fol/folScope2}

The following examples show how the primed connectors 
and quantifiers behave with respect to convertibility and 
input/output.



 \inputsnippets{FolExamples/toyNotationForm3}


\inputsnippets{FolExamples/toyNotationForm4}









\section{Computing and reasoning on first-order formulas}


\subsection{Structural recursion on formulas}

Structural induction/recursion principles are generated by 
\texttt{Formula}'s definition, for instance:

\inputsnippets{FolExamples/FormulaRect}

\subsubsection{Free variables}
The functions \texttt{freeVarT} [resp. \texttt{freeVarTs},  and
\texttt{freeVarF}] compute the multiset (as a list with possible repetitions) of the free occurrences of variables in a term [resp. a vector of terms, a formula].

\inputsnippets{folProp/freeVarT}

\inputsnippets{folProp/freeVarF, FolExamples/freeVarEx}


\begin{remark}
  Note that \texttt{freeVarF} is defined by cases over the basic connectives. In the following example, the connective \texttt{<->} is expanded into a conjunction of two implications, thus the 
list returned by \texttt{freeVarF} may contain redundancies.


\inputsnippets{FolExamples/freeVarDup}

\end{remark}

\subsubsection{Closing a formula}
Function \texttt{freeVarF} is used in 
the  function \texttt{close}, which  universally quantifies  all the free variables of a formula.

\noindent From \href{../theories/html/hydras.Ackermann.folProp.html}{Ackermann.folProp}

\inputsnippets{folProp/closeDef}

\noindent From \href{../theories/html/hydras.MoreAck.FolExamples.html}{MoreAck.FolExamples}

\inputsnippets{FolExamples/closeExample}

% \subsubsection{Computing a fresh variable}

% It is easy to get a new variable, which 

\subsection{Decidability of equality}
% Deciding whether two formulas are equal is a part of the verification of a proof (for instance in the implementation of the \emph{assumption} rule).

Let $L$ be a language, and let us assume that equality 
of function and relation symbols of $L$ are decidable.
Under this assumption, equality of terms and formulas over $L$ is decidable too.

Because of dependent types, the proofs are quite long and technical. The reader may consult them in \href{../theories/html/hydras.Ackermann.fol.html}{Ackermann.fol}



\inputsnippets{fol/formDec1, fol/formDec2,
fol/formDec3,fol/formDec4,fol/formDec5}


\begin{remark}
Please note that \texttt{term\_dec}, \texttt{terms\_dec} and
\texttt{formula\_dec} are \emph{opaque}. 

The function \texttt{formula\_dec} is mainly used in
\href{../theories/html/hydras.Ackermann.PA.html}{Ackermann.PA}, in order to check whether a given formula belongs to the axioms of Peano arithmetic.
\begin{todo}
  Look for the use of \texttt{open}  (in codePA)
\end{todo}

\end{remark}


\subsection{Variables and substitutions}

Since free and bound occurrences of a variable $i$ are represented the same way, much care should be taken in programming the substitution of a term to a variable in order to avoid \emph{variable capture}.

\begin{todo}
  Present \texttt{substT} and \texttt{substTs} as structurally recursive functions. Then show \texttt{BadSubst.v} and motivate
 depth induction.
\end{todo}


\inputsnippets{folProp/subsTdef}

\inputsnippets{BadSubst/BadSubstFdef}

\inputsnippets{BadSubst/BadSubstFexample}


For instance let $F$ be the formula 
$\exists v_2, v_1 \not= f(v_2)$. The result of the substitution 
of the term $f(v_2)$ to the (free) occurrences of $v_1$ in $F$
 \emph{is not} $\exists v_2, f(v_2) \not= f(v_2)$, but, for instance $\exists v_3, f(v_2) \not= f(v_3)$, where $v_3$ is a ``fresh'' variable.


\inputsnippets{FolExamples/substExample1}

Thus, the function (\texttt{fun f => substF L f i t}) doesn't commute with all \texttt{Formula}'s constructors.

\inputsnippets{FolExamples/substExample2}

From this little example, we can guess that the function
\texttt{substF} which is assumed to compute the substitution in a formula $F$ the substitution of a term $t$ to the free occurrences of a variable $v$ will not be implemented as a direct structural recursive function. For instance, the formula $G$, obtained through a variable renaming, is not a strict subterm of $F$.

Thus, we cannot expect do define simply this operation by 
structural recursion over formulas. 
In~\cite{Goedel}, the approach is to define a \emph{measure}
which maps every formula to a natural number, then define substitution in such a way that the substitution of a term to a free variable in a strict sub-formula  of a formula $F$ is smaller
(w.r.t. this measure) than $F$. 
Then substitution could be defined by well-founded recursion.

Fortunately, such a measure : the \emph{depth} of a formula,
can be easily defined by structural recursion.


% To fix this issue, we can associate  a measure (\emph{e.g.} a natural number) to each formula, such that the renaming of a strict subformula $F$ is strictly smaller that $F$. 



\subsubsection{Depth of a formula}

The function \texttt{depth} computes the \emph{depth} of the 
 skeleton of any formula, and allows us to define a well-founded strict order on 
\texttt{Formula $L$}.

\inputsnippets{fol/depthDef}

\begin{remark}
  The depth of a formula takes into account its abstract syntax tree \emph{with respect to the base connective and quantifiers : $\arrow$, $\sim$ and $\forall$}.
    Formulas which contain $\vee$, $\wedge$, $\exists$, etc. are translated into
    basic formulas before the computation of their depth. In the example below, 
the conjunction is translated into a bigger term than the disjunction.


\inputsnippets{FolExamples/DepthCompute}
\end{remark}

\subsubsection{Induction on depth}




% \begin{todo}
% Motivate the induction principles based on depth. Compatible with term substitution and universal quantifier elimination.
% \end{todo}



\inputsnippets{FolExamples/depthRecDemo}

\begin{todo}
 Look for the principles which are really used in Ackermann or/and Goedel libraries, and comment them.
 Maybe skip the helpers (unused in other files)
\end{todo}

The library~\href{../theories/html/hydras.Ackermann.fol.html}{Ackermann.fol} contains several induction principles, applied 
throughout \texttt{Ackermann} and \texttt{Goedel} projects.


Let us for instance have a look  at \texttt{Formula\_depth\_ind2}. Its application in order to prove $P\;a$ generates 5 sub-goals. 

\inputsnippets{FolExamples/depthRecDemo2}

\begin{itemize}
\item Goals $1$ to $4$ correspond to  usual proofs by structural induction.
\item Goal $5$ is associated with a universal quantification $f=\forall\,v,a$. In this case, we have to prove that $P\;b$ holds for any formula $b$ which has a depth strictly less than $f$. Such a $b$ may for instance the result of replacing the free occurrences of $v$ in $a$ with any term $t$.
  \begin{todo}
   Make a link to an appropriate example.
  \end{todo}
\end{itemize}



\subsection{Free variables and substitutions}



\subsection{Multiple substitutions}

\begin{todo}
Link to subAll.v
\end{todo}

\begin{todo}
Make a chapter!
\end{todo}
\section{Languages for Arithmetic}



Two languages built with the usual symbols of arithmetic are 
defined in ~\href{../theories/html/hydras.Ackermann.Languages.html}{Ackermann.Languages}.

\begin{itemize}
\item The first language: \texttt{LNT} (\emph{Language of Number Theory}) has just function symbols for $+$, $\times$, $0$ and successor.
\item The second language: LNN (\emph{Language of Natural Numbers})  has
the same function symbols as LNT plus one relation symbol for the strict inequality $<$ : \texttt{LT} (less than).
\end{itemize}

\subsubsection{Language of Number Theory (LNT)}

First, we declare two alphabets.

\inputsnippets{Languages/LNTDef1} 

In a second time, we build \texttt{LNT} and \texttt{LNN} by filling \texttt{Language}'s \texttt{arity} field.

\inputsnippets{Languages/LNTDef2} 

\begin{remark}
  We depart a little from \cite{Goedel}'s notations, where the 
function and relation symbols are called \texttt{Plus}, 
\texttt{Mult}, \texttt{LT}, etc. In our version, these type constructors are called \texttt{Plus\_}, 
\texttt{Mult\_}, \texttt{LT\_}, etc., while the names without final underscores are bound to term building functions (\emph{e.g.}
the function which takes two terms and builds the term representing their sum) (see Remark~\ref{rem:underscores}).
\end{remark}


\subsubsection{Language of Natural Numbers (LNN)}

\texttt{LNN} is an extension of \texttt{LNT}, by the addition 
of the $<$ relation symbol.


\inputsnippets{Languages/LNNDef}



\subsubsection{Examples}

Let us show a few examples (from ~\href{../theories/html/hydras.MoreAck.FolExamples.html}{MoreAck.FolExamples}).

\inputsnippets{FolExamples/arityTest} 
 



% \subsubsection{Examples}

% With our toy language, we can build the following terms, respectively written $a$, $f(a)$ and $f(f(v_1))$ in usual mathematical notation. 

% Writing these terms in bare \gallina syntax would result in quite clumsy scripts. Thus, we may use a still experimental notation scope which allows us to write/print terms and formulas in a 
% notation similar to \coq's.





% We put both the associated \coq{} terms, in bare \gallina{} syntax, and with a still experimental and provisional notation, defined in
% ~\href{../theories/html/hydras.Ackerm.FOL_notations.html}{MoreAck.FOL\_notations}).



% \inputsnippets{FolExamples/toyNotation}

% \inputsnippets{FolExamples/smallTerms}


For instance the term $v_1+0$, where $v_1$ is a variable,
is represented by the following \gallina term of type 
(\texttt{fol.Term LNT}).

\inputsnippets{FolExamples/v1Plus0} 






\inputsnippets{LNN/instantiations}

\section{Notations for Formulas (experimental)}

In order to get more readable terms and formulas, we can define a few notations in ~\href{../theories/html/hydras.MoreAck.FOL_notations.html}{MoreAck.FOL\_notations} and
~\href{../theories/html/hydras.MoreAck.LNN.html}{MoreAck.LNN}.
Please note that these notation scopes are experimental: We are going to use them in examples and exercises before using them in large original proof scripts (in the \texttt{ordinals/Ackermann/} sub-directory).

We try to define notation scopes as close as possible to \coq's syntax for propositions.

Let us take for instance the following proposition (in math form):

$$\forall\, v_0,\, v_0=0\vee \exists\,v_1,\,v_1=1+v_0$$

Here is a definition, using directly the \texttt{goedel/Ackermann}'s project syntax.

\inputsnippets{LNN_Examples/uglyF0}

Note that, because of redefinitions, the disjonction \texttt{orH}
can be expanded in terms of  implication and negation (for instance when we use \texttt{Compute}).

\begin{todo}
  Present the general issue about evaluation, and our provisional solution.
\end{todo}


\inputsnippets{LNN_Examples/CNNF0}




\section{Proofs}

\subsection{Contexts as lists}
\begin{todo}
Describe the \texttt{Prf} type (from \texttt{folProof.v}).

Give examples in a toy language.

Give an example which motivates \texttt{SysPrf}:

$P,P , P\arrow P \arrow Q, \vdash Q$ ?
\end{todo}

\begin{exercise}

Prove the following lemma:
\inputsnippets{FolExamples/eqRefl}

\end{exercise}

\subsection{Contexts as sets}

\begin{exercise}

Prove the following lemma:
\inputsnippets{FolExamples/MPdiag}

\end{exercise}


\section{Derived rules and natural deduction}


 The library 
 \href{../theories/html/hydras.Ackermann.folLogic.html}{Ackermann.folLogic} contains many derived rules which allow the user to build proofs in a natural deduction style (with introduction and elimination rules).

\subsection{Example}

For instance, let us prove Peirce's rule.
 (in  ~\href{../theories/html/hydras.MoreAck.FolExamples.html}{MoreAck.FolExamples} ).

\subsubsection{Prelude}

%\inputsnippets{folPeirce/prelude}


\subsection{Proof of Peirce's law}


\inputsnippets{FolExamples/PeirceProof}

Let us try to do an implication introduction.

\inputsnippets{FolExamples/step1}

Now, we may use the law of excluded middle with the formula $P$. The only non-trivial case is about $\sim P$.

\inputsnippets{FolExamples/step2}

The rest of the proof is composed of basic proof steps, 
and bookkeeping steps (about \texttt{Ensembles.In}).

\inputsnippets{FolExamples/step3, FolExamples/step4}

\index{ackermann}{Exercises}
\begin{exercise}
Prove, using the rules described in 
 \href{../theories/html/hydras.Ackermann.folLogic.html}{Ackermann.folLogic}, the famous \emph{drinkers theorem}:

$$\exists\,x,\; (D(x)\Longrightarrow \forall\,y,\; D(y))$$
\end{exercise}

where $D$ (for ``drinks'') is some predicate symbol of arity $1$.




\include{chapter-encoding}


\part{A few  certified algorithms}

\include{chapter-powers}

\part{Appendices}

\bibliographystyle{alpha}
\bibliography{thebib}





\chapter{Index and tables}

% \begin{todo}
%  Still very incomplete!
% \end{todo}
{\Large \textbf{In progress} This index is currently under reorganization.  We apologize for its incompleteness! }

\printindex{gaiabridge}{Links to \gaia Library}
\label{gaia-index}
\printindex{coq}{Coq, plug-ins and standard library}
\printindex{maths}{Mathematical notions and algorithmics}
\printindex{hydras}{Library hydras: Ordinals and hydra battles}
\printindex{ackermann}{Library hydras.Ackermann: Primitive recursive functions, G\"{o}del encoding}
\printindex{additions}{Library additions: Addition chains}

% \section{Main notations}






% \begin{table}[H]
%   \centering
%   \begin{threeparttable}
%     \caption{Ordinals and ordinal notations}
% \begin{tabular}{|r | c|c|c|c|l|}
% \hline
% Name & Gallina&Math& Description& Page \\\hline
% \texttt{lt : T1->T1->Prop}& lt alpha beta & $\alpha < \beta$& strict order on type \texttt{T1} \tnote{1} & \pageref{Predicates:lt-T1}\\
% \texttt{LT: T1->T1->Prop}& alpha o< beta & $\alpha < \beta$& strict order on type \texttt{T1}   \tnote{2} & \pageref{Predicates:LT-T1}\\
% \texttt{Lt : E0->E0->Prop} & alpha o< beta & $\alpha < \beta$& strict order on type \texttt{E0} \tnote{3} & \pageref{Predicates:Lt-E0} \\
% \texttt{nf: T1->Prop} & \texttt{nf alpha} && alpha is in Cantor normal form & \pageref{Predicates:nf-T1}\\
%  \texttt{on\_lt} & \texttt{alpha o< beta}&$\alpha<\beta$& ordinal inequality \tnote{4} & \pageref{sect:on-lt-notation}\\
%  \texttt{on\_le} & \texttt{alpha o<= beta}&$\alpha\leq\beta$& ordinal inequality & \pageref{sect:on-lt-notation}\\
% \texttt{plus} & \texttt{alpha + beta} & $\alpha + \beta$ & ordinal addition & \pageref{sect:infix-plus-T1}, \dots\\
% \texttt{oplus} & \texttt{alpha o+ beta} & $\alpha \oplus \beta$ & Hessenberg sum & \pageref{sect:infix-oplus} \\

% F & \texttt{F $n$} & $n$ & The $n$-th finite ordinal &  
% \pageref{sect:notation-F}, \pageref{sect:notation-F-sch}\\ 
% FS & \texttt{FS $n$} & $n+1$ & The $n+1$-th finite ordinal  \tnote{5} &  
% \pageref{sect:notation-FS}\\ 
% omega & \texttt{omega} & $\omega$ &   the first infinite ordinal   & \pageref{sect:notation-omega}, \pageref{sect:omega-T1}, \pageref{sect:omega-notation2}, \dots\\
% phi0     & \texttt{phi0 alpha} & $\phi_0(\alpha),\; \omega^\alpha$&exponential of base $\omega$ & \pageref{sect:notation-phi0}\\

% \hline
% \end{tabular}
% \begin{tablenotes}
%   \item[1] This order is total, but not well-founded, because of not well formed terms.
% \item[2] Restriction of \texttt{lt} to terms in normal form; this order is partial, but well-founded.
% \item[3] This order is total \emph{and} well-founded.
% \item [4]
% Some notations may belong to several scopes. For instance, ``\texttt{o<}'' is
% bound in \texttt{ON\_scope}, \texttt{E0\_scope}, \texttt{t1\_scope}, etc., and locally in several libraries.
%   \item [5] Note that there exist also various coercions from \texttt{nat} to types of ordinal. Depending on the current scope and  \coq's syntactic analysis algorithm, \texttt{F} may be left implicit.
% \end{tablenotes}
% \end{threeparttable}
% \end{table}



% \vspace{4pt}


% \begin{table}[H]

%   \begin{threeparttable}
%     \caption{hydra battles}
% \begin{tabular}{|c|c|c|c|l|}
% \hline
% Name & Gallina&Math& Description& Page \\\hline
% \texttt{round} & \texttt{h -1-> h'} & & one round of a battle & \pageref{sect:infix-round} \\
% \texttt{rounds} & \texttt{h -+-> h'} & & one or more  rounds of a battle & \pageref{sect:infix-rounds} \\
% \texttt{round\_star} & \texttt{h -*-> h'} & & any number of rounds of a battle & \pageref{sect:infix-rounds} \\
% \hline
% \end{tabular}

% %\begin{tablenotes}
% %\end{tablenotes}

%   \end{threeparttable}
 
% \end{table}

% \begin{table}[H]
%   \centering
%   \begin{threeparttable}
%     \caption{Addition chains}
% \begin{tabular}{|c|c|c|c|l|}
% \hline
% Name & Gallina&Math& Description& Page \\\hline
% \texttt{Mult} & \texttt{$z$ <--- $x$ times $y$} & & monadic notation & \pageref{monadic-mult} \\
% \hline
% \end{tabular}

% %\begin{tablenotes}
% %\end{tablenotes}

%   \end{threeparttable}
 
% \end{table}

\end{document} 
